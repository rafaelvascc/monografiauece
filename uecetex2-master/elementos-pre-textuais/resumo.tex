A escolha das tecnologias a serem utilizadas e das plataformas alvo é o primeiro passo na construção de um \est{software}. Sistemas escritos para a plataforma \est{web} podem ser escritos em diversas linguagens e utilizar uma variedade de \est{frameworks} e técnicas. É comum presenciar discussões entre desenvolvedores sobre qual a melhor tecnologia para se fazer um sistema, onde junto com argumentos técnicos e objetivos também são apresentados argumentos subjetivos embasados apenas na preferencia do desenvolvedor. Entre as tecnologias de desenvolvimento para \est{web} onde se encontra grande rivalidade entre seus utilizadores, e discussões sobres seus prós e contras, então as tecnologias que utilizam a linguagem de programação \lang{Java} da empresa \est{Oracle} e o \est{.NET Framework} da \est{Microsoft}. Nessa monografia, é demonstrada a criação de um \est{software} simples para \est{web} utilizando as duas tecnologias, com exemplos de códigos postos lado a lado e ressaltando as vantagens e desvantagens que uma tecnologia possui em relação a outra. No projeto feito com a linguagem Java, são utilizados os \est{frameworks} \est{Spring} e \est{Hibernate}, o projeto que utiliza o \est{framework} \est{.NET}, é escrito na linguagem C\# e utiliza o \est{ASP.NET MVC 5} e o \est{Entity Framework 6}. De modo geral, o uso da tecnologia \est{.NET} se mostrou mais prática para o desenvolvedor, sendo possível obter maior produtividade devido à menor quantidade de configuração necessária na criação de um projeto características na linguagem C\# que tornam a escrita de código mais fácil. O uso das tecnologias baseadas no Java se mostraram um pouco mais complexas, principalmente por causa da maior quantidade de configuração necessária. No entanto,  ela permite ao desenvolvedor obter um maior controle sobre o projeto. Finalmente, se o \est{software} for ser executado em um servidor \est{Windows}, a tecnologia \est{.NET} se mostrou uma melhor opção. No entanto, a tecnologia da \est{Microsoft} não possui suporte oficial para \est{Linux} ou \est{MacOS} no presente momento,  fazendo com que a tecnologia \lang{Java} seja a melhor opção para \est{softwares} que executam em múltiplas plataformas.

% Separe as palavras-chave por ponto
\palavraschave{Desenvolvimento Web. .NET. Java. Spring Framework. Entity Framework. Hibernate. Windows. Linux. Microsoft. Oracle.}