A escolha das tecnologias a serem utilizadas e das plataformas alvo é o primeiro passo na construção de um \est{software}. Sistemas escritos para a plataforma \est{web} podem ser escritos em diversas linguagens e utilizar uma variedade de \est{frameworks} e técnicas. É comum presenciar discussões entre desenvolvedores sobre qual a melhor tecnologia para se fazer um sistema, nas quais junto com argumentos técnicos e objetivos também são apresentados argumentos subjetivos embasados apenas na preferência do desenvolvedor. Entre as tecnologias de desenvolvimento para \est{web} onde se encontra grande rivalidade entre seus utilizadores, e discussões sobres seus prós e contras, então as tecnologias que utilizam a linguagem de programação \lang{Java} da empresa \est{Oracle} e o \est{.NET Framework} da \est{Microsoft}. Nessa monografia, é demonstrada a criação de um \est{software} simples para \est{web} utilizando as duas tecnologias, com exemplos de códigos postos lado à lado e ressaltando as vantagens e desvantagens que uma tecnologia possui em relação à outra. No projeto desenvolvido com a linguagem Java são utilizados os \est{frameworks} \est{Spring} e \est{Hibernate}. O projeto que utiliza o \est{framework} \est{.NET} é escrito na linguagem C\# e utiliza o \est{ASP.NET MVC 5} e o \est{Entity Framework 6}.

% Separe as palavras-chave por ponto
\palavraschave{Desenvolvimento Web. .NET. Java. Spring Framework. Entity Framework. Hibernate. Windows. Linux. Microsoft. Oracle.}