The choice of the technologies to be used and the target platforms, is the first step taken when building a software. Systems for the web platform  can be written in many languages and use many frameworks and techniques. It is common to witness arguments between developers about which is the best technology to build a system, along with technical and objective arguments, are presented subjective ones, based on the developers preference. Among technologies for web development where is found great rivalry between their users, and discussions about their pros and cons, are the technologies that uses Oracle's Java programming language and Microsoft's .NET framework. This monograph presents the creation of a simple software for the web using both technologies, with code samples side by side, highlighting advantages and disadvantages one technology has on the other. The project written in the Java language uses the frameworks Spring and Hibernate. The project that uses the .NET Framework is written with the language C\# and uses \est{ASP.NET MVC 5} and \est{Entity Framework 6}. Overall, the use of the .NET technology proved more practical to the developer, giving more productivity due to the need of fewer configuration and features of the C\# language that make coding easier. The use of Java based technologies proved to be more complex, due to the greater quantity of configuration needed for its use. But, it gives the developer a greater control on project configuration. In the end, if the software will be run on a Windows based server, the use of the .NET technologies proved to be a better option. But, this technology does not have official support for Linux or MacOS in the present time. So, for non Windows servers enviroments, the use of Java based technologies is the best option between the two.

% Separe as palavras-chave por ponto
\keywords{Web Development. .NET. Java. Spring Framework. Entity Framework. Hibernate. Windows. Linux. Microsoft. Oracle.}