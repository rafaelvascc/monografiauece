\chapter{Introdução}
\label{cap:introducao}

\section{Objetivo}
\label{sec:objetivo}

Este trabalho tem como objetivo comparar a criação de uma aplicação utilizando duas tecnologias, \textit{Java} com o \lib{Spring Framework} e \lib{Hibernate}, e o \sigla{ASP.NET MVC 5} e \textit{Entity Framework} utilizando \lang{C\#} e encontrar as vantagens e desvantagens que uma tem sobre a outra. 
É utilizado como exemplo a criação de um projeto de uma aplicação \textit{web} de três camadas (apresentação, serviços e persistência), que utiliza o padrão \est{Model View Controller} (\sigla{MVC}).

A abordagem utilizada consiste em dividir as tarefas necessárias para a criação da aplicação em cinco capítulos, e nesses capítulos serão demonstrados exemplos de código com as duas tecnologias. 
Depois de expostos os exemplos de como executar as tarefas, serão ressaltadas as vantagens, desvantagens e características distintas de cada tecnologia. As tarefas a serem executadas serão as seguintes:

\begin{itemize}
  \item Preparação do ambiente de desenvolvimento.
  \item Criação e configuração de um novo projeto.
  \item Criação de \est{controllers}.
  \item Criação de \est{views}.
  \item Criação de classes de serviço.
  \item Configuração de injeção de dependências.
  \item Criação de entidades.
  \item Criação de repositórios.
\end{itemize}

\subsection{Aplicação de exemplo}

A aplicação de exemplo consiste em 3 páginas \textit{web}. 
Uma recebe uma lista de objetos do banco de dados e os mostra em uma tabela, como mostra a Figura~\ref{fig:listpage}. 
Essa página possui \textit{links} para a criação, edição e deleção de objetos. 

\figura{listpage.jpg}{Página de lista de objetos}{fig:listpage}

Os \textit{links} de adição e edição levam às suas respectivas páginas, exibidas pela Figura~\ref{fig:addedit}. 
Essas páginas são semelhantes em aparência, no entanto a página de adição é carregada com todos os seus campos vazios e os campos da página de edição são carregados com os valores do objeto que se está editando. 
Além disso, o botão \textit{save} em cada página chama uma ação diferente no servidor.
Quando o usuário salva as modificações feitas em ambas as páginas, o sistema retorna para página de lista, que exibe as informações atualizadas.
O \textit{link} de deleção apaga o objeto do banco de dados e recarrega a página de lista para atualizar as informações exibidas.

\figura{addedit.jpg}{Paginas de adição e edição de objetos}{fig:addedit}

\section{Estado da arte}
\label{sec:estadoarte}
 
A literatura conta com vários trabalhos sobre medição de desempenho de aplicações \est{web} e poucos comparando como se cria uma aplicação. 
Mesmos trabalhos comparando o esforço do desenvolvimento de aplicações utilizando tecnologias diferentes, utilizam versões antigas de tais tecnologias. 

O artigo \est{"Critical Comparison Of PHP And ASP.NET For Web Development"}~\cite{14}, por exemplo, compara a criação de um site utilizando uma versão antiga do \textit{ASP.NET} e utilizando \textit{PHP}. 
Nesse trabalho não são expostos exemplos de código, apenas as conclusões tiradas pelo autor. 
Esse artigo também não detalha quais \textit{frameworks} foram utilizados na construção do \textit{software}. 
É dito que se usa o \textit{ASP.NET} mas não é especificado se é utilizado o \textit{ASP.NET MVC} ou \textit{ASP.NET WebForms}, e nem qual versão é utilizada. 

Oo artigo "\est{"JSF vs ASP.NET, what are their limits?"}~\cite{15} também comparam o desenvolvimento de aplicações \textit{web} utilizando as tecnologias \textit{Java} e \textit{.NET}. 
Mas também utilizam versões antigas de ambas as tecnologias e demonstra poucos exemplos de código. Esse trabalho compara o uso do \textit{ASP.NET WebForms} com o \textit{Java Server Faces} (\est{JSF}) 
com um foco maior em explicar o funcionamento interno das tecnologias em vez de mostrar como utiliza-las.

Tiago Bencardino, em seu trabalho de conclusão do curso de engenharia de teleinformática de título "\est{iQuizzer}: Integrando aplicações web e dispositivos móveis em um ambiente para criação e execução de \est{quizzes}"~\cite{16}, 
comparou o desenvolvimento de uma aplicação para as plataformas moveis \textit{Android} e \textit{iOS}. Bencardino utilizou uma quantidade maior de exemplos de código, comparando o desenvolvimento das mesmas tarefas em ambas as plataformas alvos de seu trabalho.

A Tabela 1 contém um resumo do conteúdo dos trabalhos citados:

\begin{table}[h!]   
    \centering
    \Caption{\label{tab:tbl1} Conteúdo de trabalhos sobre comparação de tecnologias.}
    \UECEtab{}{
        \begin{tabular}{ccll}
            \toprule
                Autor/trabalho & Tecnologia recente & Exemplos & Tecnologia \est{Web} \\
            \midrule \midrule
                Atul Mishra & Não & Não & Sim \\
                Mostafa Pordel e Faranhaz Yekeh & Não & Poucos & Sim \\
                Tiago Bencardino & Sim & Sim & Não \\
								Este trabalho & Sim & Sim & Sim \\
            \bottomrule
        \end{tabular}
    }{
        \Fonte{Elaborado pelo autor}
    }
\end{table}

\section{Resumo dos resultados alcançados}
\label{sec:resumoresultados}

Ambas as tecnologias possuem seus pontos fortes e fracos. De modo geral, o uso de \textit{Java} com o \textit{Spring Framework} proporciona melhor controle do projeto ao desenvolvedor e a liberdade de utilizar uma gama maior de bibliotecas para sua configuração. 
Mas essa liberdade tem um custo, o desenvolvedor precisa escrever mais código de configuração e pesquisar como os componentes escolhidos interagem para que o projeto seja bem sucedido.

A tecnologia \textit{ASP.NET MVC} é mais fechada (apesar de ter seu código ser aberto), existem muitas convenções que o desenvolvedor deve obedecer e não existe tanta liberdade para configuração pelo \textit{framework}. 
A sua vantagem é que o desenvolvedor se preocupa menos em como vai configurar sua aplicação, podendo utilizar seu tempo para escrever código que realmente agrega valor ao negócio da aplicação.

De modo geral, se o desenvolvedor dispuser de tempo e precisar de um controle maior sobre o projeto, o uso da tecnologia \textit{Java} é aconselhável. Por outro lado, se o projeto tiver um prazo reduzido ou o desenvolvedor preferir seguir convenções e não quiser se preocupar em como vai ter que configurar o projeto, a tecnologia \textit{.NET} pode ser uma melhor opção.
 
\section{Publico alvo}
\label{sec:publicalvo}

Esse trabalho pode beneficiar gerentes de projetos e lideres técnicos à escolher qual das duas tecnologias utilizar para iniciar um novo projeto. Estudantes interessados em começar a desenvolver para a \textit{web} também podem se beneficiar desse trabalho e aprender o básico sobre as duas tecnologias.

\section{Metodologia de pesquisa}
\label{sec:metodologiapesquisa}

A maior fonte da informação para esse trabalho foram livros, em sua maioria as edições mais recentes. Além dos livros, foram utilizados como fonte de pesquisa as documentações online de ambas as tecnologias. Também foram utilizados como fonte de pesquisa, exemplos encontrados em sites de desenvolvedores na internet, sendo o mais utilizado o \textit{stack overflow} americano.