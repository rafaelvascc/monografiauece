\chapter{Classes de serviços e injeção de dependências}

Uma aplicação de três camadas típica se divide em camadas de apresentação, negócios/serviços e persistência de dados. A camada de serviços é onde se escreve o código que representa regras de negócio das aplicações, enquanto a camada de persistência é responsável por guardar os dados da aplicação para uso posterior. É importante que essas camadas estejam fracamente acopladas, pois o desenvolvedor pode, por exemplo, aproveitar as classes de negócio que ele escreveu para uma aplicação \textit{web} em uma aplicação para celular.

Nos exemplos desse trabalho, a camada de apresentação é formada pelas páginas \textit{web} escritas no capítulo anterior. Os \textit{controllers} funcionam como uma ponte entre a camada de apresentação e de negócios \cite{6}. \textit{Controllers} devem apenas receber e responder requisições, o ideal é que não exista código de regras de negócios em suas ações.

Nesse capítulo é demonstrado como escrever classes de serviço (também chamadas de classes de negócio) fracamente acopladas à camada de apresentação e aos \textit{controllers}. 

\section{Classes de serviços}

É criada como exemplo uma classe de serviço em cada projeto que contêm um método com uma regra de negócio simples. As listas de objetos \textit{Person}, até agora pertencente aos \textit{controllers}, são movidas para essas classes como uma forma de representar a camada de persistência. Em ambos os projetos, o serviço implementa uma \textit{interface}. Diferente do que é sugerido para \textit{controllers} (um \textit{controller} não deve chamar métodos de outro) classes de serviço podem utilizar outras classes de serviço.

\subsection{\textit{Java}}

No projeto \textit{Java}, a \textit{interface} e classe concreta são criadas no pacote \textit{br.uece.webCrud.service}. Primeiro é feita a interface \textit{PersonService} que o serviço irá implementar, como se pode observar no código fonte~\ref{lst:22}.

\javacode{code/22.txt}{\textit{Interface} \textit{PersonService}}{lst:22}

No código fonte~\ref{lst:23} pode-se observar o serviço \textit{PersonServiceImpl} implementando a \textit{interface} \textit{PersonService}.

\javacode{code/23.txt}{Classe \textit{PersonServiceImpl}}{lst:23}

A classe \textit{PersonServiceImpl} começa decorada com a anotação \textit{@Service} (linha 7). Essa anotação é usada pelo \textit{Spring Framework} para marcar classes da camada de serviços. A anotação \textit{@Service} herda da anotação \textit{@Component}, então essa é uma classe que pode ser usada pelo \textit{Spring IoC Container} durante a injeção de dependências. Mais detalhes sobre injeção de dependências e o modo como o \textit{Spring} injeta uma instância dessa classe serão explicados nas seções 6.2 e 6.2.1. Segundo a documentação do \textit{Spring Framework 4.1.4}, a anotação \textit{@Service} tem o mesmo efeito que \textit{@Component}, mas é recomendado o uso de \textit{@Service}, pois em versões futuras do \textit{framework} essa anotação poderá conter funcionalidades adicionais.

O construtor de \textit{PersonServiceImpl} inicializa a lista de objetos \textit{Person}. O método \textit{add} recebe um objeto do tipo \textit{Person} como parâmetro. Ele verifica se já existe um objeto com o mesmo nome na lista, se sim, dispara uma exceção, se não, adiciona o objeto à lista. Essa verificação é um exemplo simples de uma regra de negócios.

\subsection{\textit{.NET}}

As classes de serviços no projeto \textit{ASP.NET MVC} são criadas dentro da pasta \textit{Models/Services} (ver figura~\ref{fig:19}). Para o \textit{ASP.NET MVC} a camada \textit{model} de uma aplicação \textit{MVC} contém tanto entidades quanto regras de negócios. Assim como no projeto \textit{Java}, é feita a \textit{interface} \textit{IPersonService} mostrada no código fonte~\ref{lst:24}.

\sharpcode{code/24.txt}{\textit{Interface} \textit{IPersonService}}{lst:24}

O código fonte~\ref{lst:25} mostra a implementação da \textit{interface} pela classe \textit{PersonService}.

\sharpcode{code/25.txt}{Classe \textit{PersonService}}{lst:25}

A classe \textit{PersonService} no projeto \textit{ASP.NET MVC} é uma classe comum para onde foi movida a lista de objetos \textit{Person}. Essa classe contém um método idêntico ao do projeto \textit{Java}. Nenhuma anotação ou configuração especial é necessária para identificar a classe como um serviço.

\section{Injeção de dependências}

Injeção de dependências é uma técnica utilizada para diminuir o acoplamento entre classes \cite{8}. Nos exemplos desta seção, as classes de serviço são injetadas nos \textit{controllers} criados no capítulo anterior. 

Considere o exemplo do código fonte~\ref{lst:26}.

\javacode{code/26.txt}{Exemplo de classes fortemente acopladas}{lst:26}

No exemplo acima, a classe \textit{Foo} (linha 12) usa uma classe chamada \textit{BarImpl} (linha 7) que implementa uma \textit{interface} chamada \textit{Bar} (linha 3). Nesse exemplo a classe \textit{Foo} fica fortemente acoplada à implementação \textit{BarImpl}. Imagine se a classe \textit{BarImpl} fosse uma classe de persistência que salva informações em arquivos de texto e um dia o desenvolvedor precisasse portar a aplicação para mais um ambiente onde seria usado um banco de dados. Sem usar injeção de dependências, o desenvolvedor também teria que escrever uma nova versão da classe \textit{Foo} que usasse outra implementação da \textit{interface} \textit{Bar}.

O exemplo do código fonte~\ref{lst:27} mostra como funciona a injeção de dependências.

\javacode{code/27.txt}{Exemplo de classes fracamente acopladas}{lst:27}

No exemplo acima, a classe \textit{Foo} agora recebe em seu construtor um abjeto que implemente \textit{Bar} (linha 3). Uma instância de \textit{BarImpl} é criada dentro do método \textit{main} (linhas 12) e injetada dentro da instância de \textit{Foo} usando seu construtor (linha 13). Nesse cenário, as classes não estão tão acopladas e o desenvolvedor pode usar outras implementações de \textit{Bar} mais facilmente. A injeção de dependências também é conhecida como inversão de controle (\textit{Inversion of Control} ou \textit{IoC}) \cite{6}, pois dá a responsabilidade de inicializar dependências de uma classe para classes externas. 

Essa técnica ainda pode ser melhorada se a injeção de dependências for automatizada, requerendo menos código. É isso que o \textit{IoC Container} do \textit{Spring Framework} e o \textit{Ninject} fazem. O uso dessas ferramentas será demonstrado nas seções seguintes.

\subsection{\textit{Java}}

O \textit{Spring IoC container} resolve dependências procurando por classes para serem injetadas dentro dos pacotes da aplicação de modo automático. Na seção 4.1.3, a classe de configuração \textit{SpringMvcConfig} foi decorada com a anotação \textit{@ComponentScan} (código fonte 5, linha 26). Os nomes dos pacotes passados como parâmetro para essa anotação são os pacotes onde o \textit{Spring Framework} procura por classes para serem instanciadas e injetadas.

Para serem utilizadas pelo \textit{Spring IoC container}, as classes devem estar decoradas com a anotação \textit{@Component} ou alguma de suas especializações, como \textit{@Service}, \textit{@Repository} e \textit{@Controller} ou métodos decorados com a anotação \textit{@Bean}. Por padrão, o \textit{Spring IoC container} injeta as classes como \textit{singletons}, criando apenas uma instância da classe para ser usada várias vezes. Esse comportamento pode ser alterado pela anotação \textit{@Scope} nas classes que se deseja injetar.

A diferença entre as anotações \textit{@Component} e \textit{@Bean} (também utilizada na seção 4.1.3) é que \textit{@Component} é usado em classes que o próprio \textit{Spring IoC container} procura e instancia enquanto \textit{@Bean} é usado em métodos que retornam objetos que o próprio desenvolvedor criou. Classes anotadas com \textit{@Component} ou uma de suas especializações devem ter um construtor que não receba parâmetros.

O código fonte~\ref{lst:28} mostra uma nova versão do \textit{controller} \textit{PersonController} do projeto \textit{Java/Spring}, agora com uma instância de \textit{PersonServiceImpl} sendo injetada e utilizada.

\javacode{code/28.txt}{\textit{PersonController} utilizando \textit{PersonService} no projeto \textit{Java/Spring}}{lst:28}

No lugar da lista em memória, agora o \textit{controller} possui e utiliza um \textit{PersonService} que  está decorado com a anotação \textit{@Autowired} (linhas 15 e 16). Essa anotação marca propriedades onde o \textit{Spring IoC container} deve injetar dependências. No caso acima, o \textit{container} procura por classes decoradas com \textit{@Component} ou suas especializações, que implementam \textit{PersonService} e estão contidas nos pacotes configurados por \textit{@ComponentScan}. Encontrada a classe \textit{PersonServiceImpl}, uma instância dessa classe é criada e atribuída à propriedade \textit{personService} para uso no \textit{PersonController}.

Essa não é a única maneira de injetar dependências usando o \textit{Spring}. É possivel, por exemplo, utilizar arquivos de configuração no formato XML para configurar a injeção de dependências. Para esse trabalho foi escolhida a técnica de injeção por anotações por ser a mais moderna e de fácil utilização. Para conhecer mais opções de injeção de dependências, aconselha-se consultar a documentação do Spring.

\subsection{\textit{.NET}}

O \textit{ASP.NET MVC 5} não possui injeção de dependências automática nativa, então é usada a biblioteca \textit{Ninject} para essa funcionalidade. Diferente do \textit{Spring}, o \textit{Ninject} não procura por classes de modo automático para injetar dependências, as classes que o desenvolvedor deseja injetar devem ser configuradas por código. Na seção 4.2.1, o \textit{Ninject MVC 5} foi adicionado ao projeto e uma classe chamada \textit{NinjectWebCommon} (exibida no quadro~\ref{lst:29}) foi criada na pasta \textit{App\_Start}. 

\sharpcode{code/29.txt}{Classe \textit{NinjectWebCommon}}{lst:29}

O método \textit{Start} é executado quando a aplicação é inicializada (linha 14). Ele registra módulos do \textit{Ninjet} para uso interno da aplicação, cria e inicializa um \textit{kernel} do \textit{Ninject}. O \textit{kernel} é o objeto onde é feita a configuração de injeções de dependência, essa configuração é feita no método \textit{RegisterServices} como mostra o quadro~\ref{lst:30}.

\sharpcode{code/30.txt}{Registro de dependências no \textit{Ninject}}{lst:30}

O \textit{kernel} do \textit{Ninject} pode usar uma sintaxe fluente para configurar dependências. Aqui usa-se o método genérico \textit{Bind} recebendo a interface \textit{IPersonService} seguido do método \textit{To} configurando a classe \textit{PersonService} para ser a classe concreta injetada quando o \textit{Ninject} encontra aquela interface. Por último, o método \textit{InRequestScope} cria uma instância de \textit{PersonService} para cada requisição do usuário ao servidor. Existe o método \textit{InSingletonScope} que pode criar apenas uma instância da classe para todos os usuários e todas as requisições, mas a necessidade de se utilizar \textit{InRequestScope} é abordada na seção 7.3.2.

O \textit{Ninject} é uma biblioteca extensa que contém módulos que funcionam em vários tipos de projetos \textit{.NET}. Para mais informações aconselha-se acessar a página do \textit{Ninject} (\url{http://www.ninject.org/}).

O \textit{controller} \textit{PersonController} é modificado para utilizar o serviço \textit{PersonService}, como pode ser observado no código fonte ~\ref{lst:31}.

\sharpcode{code/31.txt}{\textit{PersonController} do projeto \textit{ASP.NET MVC} usando \textit{PersonService}}{lst:31}

A anotação \textit{Inject} decora a propriedade \textit{PersonService} onde é injetada a instância de classe \textit{PersonService} (linha 5), do mesmo modo que \textit{@Autowired} no projeto \textit{Java/Spring}. A propriedade deve ter um método \textit{set} público para que o \textit{Ninject} possa ter acesso e injetar a classe concreta, o método \textit{get} pode ser mais restrito.

\section{Conclusão}

O \textit{Spring} possui a anotação \textit{@Service} para marcar classes de serviço e classes de negócio. Enquanto no \textit{ASP.NET MVC} qualquer classe pode ser uma classe de serviço. Atualmente, nenhuma das tecnologias mostra vantagem na criação de serviços, mas, no futuro, talvez sejam adicionadas funcionalidades para a anotação \textit{@Service} que possam beneficiar o \textit{Spring Framework}.

O \textit{Spring} é superior ao \textit{ASP.NET MVC 5} em se tratando de injeção de dependências. Além de ter um \textit{IoC container} nativo, o \textit{Spring} resolve dependências automaticamente, enquanto o \textit{ASP.NET MVC 5} precisa de bibliotecas de terceiros, e o \textit{Ninject} precisa de configurações via código para todas as classes que são injetadas.

Na próxima versão do \textit{ASP.NET}, também chamada de \textit{vNext}, será adicionado um \textit{IoC Container} nativo. Até o momento em que este trabalho foi realizado, o \textit{vNext} estava em fase de testes público e poderia passar por mudanças, então o uso do \textit{IoC Container} da próxima versão não foi considerado.

No próximo capítulo é abordada a criação do banco de dados a partir de entidades de negócio e acesso a dados usando repositórios do \textit{Spring} e a classe \textit{DbContext} do \textit{Entity Framework}.