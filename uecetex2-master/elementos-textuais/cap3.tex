\chapter{Criando projetos \textit{web} no padrão \textit{MVC}}

Nesse capitulo é mostrado como criar um novo projeto para uma aplicação \textit{web} que utiliza o padrão \textit{MVC} nas plataformas \spring e \anmvc. Para \textit{Java} serão usadas as bibliotecas \textit{Spring Framework}, que cuidará da arquitetura \textit{MVC} e injeção de dependências, e o \textit{Hibernate}, que cuidará da persistência e acesso a dados. Na plataforma \textit{.NET} é utilizado o \textit{ASP.NET MVC 5}, que cuida de toda estrutura de uma aplicação web \textit{MVC}, o \textit{Entity Framework 6} para acesso a dados e o \textit{Ninject} para injeção de dependências.

\section{Criando um projeto do \textit{Gradle} no \textit{Eclipse}}

O modelo de projeto usado nos exemplos desse trabalho é o \textit{Gradle Project}. Esse modelo de projeto está localizado na pasta \textit{Gradle} na árvore de novos projetos do \textit{Eclipse}. Na Figura~\ref{fig:gradleproject} pode-se observar a localização do modelo e a criação do novo projeto.

\figura{gradleproject1.png}{Criando um projeto do \textit{Gradle}}{fig:gradleproject}{1}

É criado um projeto com a estrutura mostrada na Figura~\ref{fig:gradletree}.

\figura{gradleproject2.png}{Estrutura inicial de um projeto do \textit{Gradle}}{fig:gradletree}{1}

As pastas \arquivo{"src/main/java"} e \arquivo{"src/main/resources"} devem armazenar, respectivamente, código fonte \textit{Java} e recursos utilizados na aplicação.  As pastas \arquivo{"src/test/java"} e \arquivo{"src/test/resources"} são utilizadas para testes unitários. As pastas de testes não serão utilizadas nos exemplos e foram removidas. A pasta \textit{build} é para onde irão todos os artefatos gerados pelo projeto.

O arquivo \arquivo{build.gradle}, exibido no código fonte~\ref{lst:1}, é o arquivo de definição de projeto do \textit{Gradle}. Nele podem ser criados \textit{scripts} para controlar a construção de artefatos, adquirir bibliotecas de terceiros, configurar testes unitários e realizar várias outras tarefas. Os \textit{scripts} do \textit{Gradle} são escritos em \textit{Groovy}, outra linguagem compatível com a máquina virtual \textit{Java}. 

\groovycode{code/buildgradle.txt}{O arquivo \textit{build.gradle}}{lst:1}

A estrutura do projeto e o arquivo \arquivo{build.gradle} gerados devem ser modificados para servir à uma aplicação \textit{web}. É adicionada uma nova pasta, chamada de \textit{WebContent}, para armazenar conteúdo especifico para \textit{web} (páginas \textit{jsp/html}, arquivos \textit{javascript} e \textit{css}). Dentro dela deve ser criada uma pasta chamada \textit{WEB-INF} para armazenar páginas \textit{jsp}. Por padrão, usuários de sistemas \textit{web} \textit{Java Enterprise Edition} não possuem acesso direto à pasta \textit{WEB-INF}, então é um lugar seguro para armazenar páginas. Na Figura~\ref{fig:12} pode ser observada a estrutura do projeto com a pasta \textit{WEB-INF}.

\figura{prrojetocomweninf.png}{Estrutura do projeto com a pasta \textit{WebContent}}{fig:12}{1}

Em seguida, é necessário modificar o arquivo \arquivo{build.gradle}. O \textit{Gradle} dispõe de vários \textit{plugins} que facilitam seu uso em diversas situações e servem a propósitos específicos. Para gerar arquivos \arquivo{.war} (artefatos de aplicações \textit{web}) e informar ao \textit{Gradle} que a pasta \textit{WebContent} armazenará o conteúdo especifico para \textit{web}, é utilizado o \textit{plugin} chamado \textit{war}. Para fazer com que o \textit{Eclipse} veja o projeto como um projeto para \textit{web} que possa ser enviado para o \textit{Tomcat}, é utilizado o \textit{plugin} \textit{eclipse-wtp}. O início do arquivo \arquivo{build.gradle} deverá ficar como no exemplo do código fonte~\ref{lst:2}.

\groovycode{code/newgradle.txt}{Arquivo \arquivo{build.gradle} com novos \textit{plugins}}{lst:2}

Com tudo devidamente configurado, é necessário atualizar o tipo de projeto para que o \textit{Eclipse} o veja como um projeto de uma aplicação \textit{web}. Pressionando Ctrl + Alt + Shift + R, o \textit{Eclipse} abrirá uma caixa de texto onde poderão ser executadas tarefas do \textit{Gradle}. Uma dessas tarefas é o comando \textit{"eclipse"} que gera arquivos adicionais para que o projeto possa ser detectado como uma aplicação \textit{web}. O comando é executado como na Figura~\ref{fig:13} e o \textit{Eclipse} agora poderá publicar a aplicação no \textit{Tomcat}.

\figura{gradlecommand.png}{\textit{Gradle Task Quick Launcher}}{fig:13}{1}

Após executar esse comando, a estrutura do projeto mudará um pouco. As pastas destinadas a conter código \textit{Java} e bibliotecas usadas no projeto serão movidas para uma pasta chamada \textit{Java Resources}.

\subsection{Adicionando o projeto ao \textit{Tomcat}}

Para adicionar uma aplicação \textit{web} ao servidor \textit{Tomcat} configurado no \textit{Eclipse}, na aba \textit{"servers"} dá-se um duplo clique no servidor para abrir suas configurações, e na aba \textit{"modules"} clica-se no botão \textit{"Add Web Module"}. Escolhido o projeto \textit{web} que se deseja adicionar, o caminho da aplicação pode ser configurado. A Figura~\ref{fig:14} ilustra esse procedimento. Com a configuração padrão do \textit{Tomcat}, a aplicação estará disponível no endereço com o seguinte formato: \url{http://<endereço-ip>:8080/<caminho-da-aplicação>}. Esse endereço é referenciado no restante do trabalho como raiz da aplicação.

Iniciando o \textit{Tomcat}, clicando nos botões \textit{"Start the server"} ou \textit{"Start the server in debug mode"} e, acessando o endereço da aplicação, é mostrado uma página de erro 404. Isso acontece porque ainda não existem as configurações do \textit{servlet} padrão, do \textit{Spring MVC} e ainda não foi criado nenhum \textit{controller} e nenhuma \textit{view}. Nesse capítulo ainda é demonstrado como configurar o \textit{servlet} padrão e o \textit{Spring MVC}. No capitulo 3 é demonstrado como se criam \textit{controllers} e \textit{views}.

\figura{tomcatproject.png}{Adicionando um projeto \textit{web} ao \textit{Tomcat}}{fig:14}{1}

\subsection{Adicionando dependências ao projeto}

Abaixo estão listadas as bibliotecas que são usadas no projeto \spring. O \textit{Gradle} pode adquirir essas bibliotecas e suas dependências do repositório central do \textit{Maven}.

\begin{itemize}
  \item \lib{Java Servlet API 3.1.0} - Biblioteca base para programar em \textit{Java} para \textit{web} usando \textit{servlets}.
  \item \lib{Spring Web MVC 4.1.4 RELEASE} - Biblioteca para criar um projeto \textit{MVC} com \textit{Spring}.
  \item \lib{Spring Transaction 4.1.4 RELEASE} - Gerenciador de transações com o banco de dados.
  \item \lib{Spring Object/Relational Mapping 4.1.4 RELEASE} - Gerenciador de entidades e mapeamento objeto/relacional.
  \item \lib{Jackson Databind 2.5.1} - Serializa e de deserializa objetos \textit{Java} no formato \textit{JSON}.
  \item \lib{MySQL Java Connector 5.1.34} - Comunicação com o banco de dados \textit{MySQL}.
  \item \lib{Hibernate JPA Support 4.3.8 Final} - Adaptador do \textit{Hibernate} para padrões \textit{Java Persistence API}.
  \item \lib{Hibernate Validator Engine 5.1.3 Final} - Biblioteca para validar dados de entidades.
  \item \lib{Hibernate c3p0 Integration 4.3.8} - Gerenciador de conexões com o banco de dados.
\end{itemize}

Os detalhes sobre as funcionalidades das bibliotecas de comunicação com o banco de dados e mapeamento objeto/relacional são abordados no capítulo 6. 

Para fazer com que o \textit{Gradle} adquira as bibliotecas necessárias e suas dependências, a seção \textit{"dependencies"} do arquivo \arquivo{build.gradle} é alterada como mostra o código fonte~\ref{lst:3}.

\groovycode{code/gradledependencies.txt}{Adicionando dependências ao projeto}{lst:3}

Com o arquivo \arquivo{build.gradle} alterado e salvo, o comando para baixar as dependências para o projeto está localizado no menu de contexto do projeto (clicando com o botão direito do \textit{mouse} sobre ele) no caminho \textit{Gradle/Refresh Dependencies}, que pode ser observado na Figura~\ref{fig:15}. O \textit{Gradle} irá adquirir todas as bibliotecas especificadas no arquivo \arquivo{build.gradle} e suas dependências automaticamente. O \textit{Spring MVC}, por exemplo, depende do \textit{Spring Core} para funcionar e o \textit{Hibernate JPA Support} precisa do \textit{Hibernate Core}. Todas as dependências do projeto ficam visíveis na seção \textit{Libraries/Gradle Dependencies}.

\figura{refreshdependencies.png}{Atualizando dependências de um projeto \textit{Gradle}}{fig:15}{1}

\subsection{Configurando a aplicação \textit{web}}

Existem duas maneiras de se configurar uma aplicação \textit{Java}, usando arquivos \textit{XML} ou escrevendo a configuração em \textit{Java}. Nos exemplos dessa seção, é usada a segunda opção.

Os arquivos de configuração são armazenados no pacote \pacote{br.uece.webCrud.config}. Dentro desse pacote são criadas duas classes, \classe{AppInitializer} e \classe{SpringMvcConfig}, como mostra a Figura~\ref{fig:16}. 

\figura{configpackage.png}{Estrutura do projeto \spring com pacote para arquivos de configuração}{fig:16}{1}

A primeira configuração que deve ser feita é usar a classe \classe{DispatcherServlet} como o \textit{servlet} padrão e configurar o contexto da aplicação (configurações especificas do \textit{Spring Framework}). \textit{Servlets} são classes \textit{Java} que processam requisições para aplicação, elas podem servir páginas para o usuário, retornar objetos \textit{JSON} e redirecionar requisições para outros \textit{servlets}. Para mais informações sobre \textit{servlets} recomenda-se o livro \textit{Murach’s Java Servlets and JSP} \cite{10}.

A classe \classe{AppInitializer} herda da interface \classe{WebApplicationInitializer} (linha 9). Para inicializar a aplicação, o \textit{Spring Framework} irá procura classes que herdam dessa \textit{interface} dentro dos pacotes Java contidos no projeto. O conteúdo da classe deve estar como no código fonte~\ref{lst:4}.

A interface \classe{WebApplicationInitializer} possui a assinatura de apenas um método, \metodo{onStartup}. Na implementação desse método é escrito código para configurar tanto o contexto da aplicação quando o \textit{servlet} primário.

Primeiro é criado um objeto do tipo \classe{AnnotationConfigWebApplicationContext} e é usado o seu método \metodo{register} para adicionar a classe \classe{SpringMvcConfig} como uma classe de configuração do \textit{Spring} (linhas 11 e 12). A classe \classe{AnnotationConfigWebApplicationContext} habilita o uso de anotações \textit{Java} para configurar demais classes, o conteúdo da classe \classe{SpringMvcConfig} é exposto mais adiante.

É criado então um \textit{servlet} do tipo \classe{DispatcherServlet} que recebe o contexto do \textit{Spring} como parâmetro (linhas 14).É usada a classe \classe{ServletRegistration} para registrar o \textit{servlet} à aplicação (linha 16). O método \metodo{addMapping} recebe como parâmetro o caractere "/", isso quer dizer que esse \textit{servlet} é usado para todas a páginas \textit{web} e demais caminhos dentro da aplicação. O método \metodo{setLoadOnStartup} recebe como parâmetro o valor 1, para configurar o \textit{servlet} como o primeiro que o servidor deve usar.

\javacode{code/AppInitializer.txt}{Classe \classe{AppInitializer}}{lst:4}

Vários objetos que fazem parte da estrutura da aplicação (também conhecidos como \textit{Spring Beans}) são configurados na classe \classe{SpringMvcConfig}, como mostra o código fonte~\ref{lst:mvcconfig}. A anotação \textit{@Bean} do \textit{Spring Framework} decora métodos na classe \textit{SpringMvcConfig} que retornam objetos que são usados no núcleo da aplicação e definem seu contexto (linhas 47, 57, 69 e 80).

A classe \classe{SpringMvcConfig} começa decorada com as seguintes anotações, das linhas 23 à 26:

\begin{itemize}
  \item \annotation{@Configuration} - Define a classe como uma classe de configuração do \textit{Spring Framework}.
  \item \annotation{@EnableWebMvc} - Habilita o \textit{Spring MVC}.
  \item \annotation{@EnableTransactionManagement} - Habilita o controle automático de transações com o banco de dados pelo \textit{Spring Framework}.
  \item \annotation{@ComponentScan} - Configura os nomes de pacotes onde o \textit{Spring IoC Container} deve procurar classes decoradas a anotação \textit{@Component} e suas especializações, como \textit{@Repository}, \textit{@Service} e \textit{@Controller} (Mais detalhes sobre essas anotações em capítulos posteriores).  
\end{itemize}

\javacode{code/mvcconfig.txt}{Classe \classe{SpringMvcConfig}}{lst:mvcconfig}

A classe começa definindo várias propriedades que são usadas para a criação da conexão com o banco de dados, como o \textit{driver} a ser utilizado, localização, nome do banco, usuário e senha (linhas 28 à 31). Outras propriedades armazenam configurações do \textit{JPA} (padrão do \textit(Java) de acesso à dados implementado pelo \textit{Hibernate}), como qual dialeto da linguagem \textit{SQL} usar para gerar consultas no banco (dialeto do \textit{MySql 5}), qual ação executar na configuração \textit{hibernate.hbm2ddl.auto} (\textit{"create"} para criar o banco de dados e gerar tabelas a partir de classes decoradas com a anotação \textit{@Entity}) e qual o esquema padrão a ser usado nas consultas (normalmente o mesmo nome do banco de dados no \textit{MySql}) (linhas 32 à 34).

A classe \textit{SpringMvcConfig} herda da classe \classe{WebMvcConfigurerAdapter} (linha 27) para ter acesso a alguns métodos que facilitam a configuração da aplicação. Esses métodos decorados com \textit{@Override} (linhas 36 e 41) configuram o uso do \classe{ServletHandler} padrão do \textit{Spring} e como a aplicação deve servir diferentes tipos de conteúdo ao usuário.

Os dois primeiros métodos decorados com \annotation{@Bean} (linhas 47 e 57) retornam objetos com informações de onde e como encontrar as \textit{views} (páginas \textit{web}) da aplicação. O \classe{InternalResourceViewResolver} (linha 48) armazena configurações sobre em que diretório estão as \textit{views} (\textit{WEB-INF}) e suas extensões (\textit{.jsp}). Assim é dito ao \textit{Spring Framework} para procurar páginas \textit{.jsp} dentro do caminho \arquivo{/WebContent/WEB-INF}. É importante lembrar neste momento que a pasta \textit{WebContent} foi configurada pelo \textit{Gradle} como pasta raiz onde deve ser armazenado todo conteúdo \textit{web} na seção 4.1. 

O Segundo método configura o \classe{ContentNegotiatorViewResolver}, uma classe de uso interno do \textit{Spring} que resolve \textit{views} baseadas no cabeçalho das requisições \textit{HTTP} (linha 58). Note que ele recebe uma coleção de \classe{ViewResolvers} como parâmetro e é adicionado o \classe{InternalResourceViewResolver} do método anterior nessa lista.

Os três últimos métodos decorados com a anotação \annotation{@Bean} retornam objetos relacionados ao uso do banco de dados e manipulação de entidades. O primeiro deles (linha 70) retorna um objeto do tipo \classe{ComboPooledDatasource} que recebe vários parâmetros para se conectar ao banco de dados e servir como fonte de dados para a aplicação. 

O segundo método (linha 81) cria um objeto do tipo \classe{LocalContainerEntityManagerFactoryBean} que gerenciará a criação de uma implementação da interface \textit{EntityManager}. Implementações de \classe{EntityManager} são usadas nos repositórios para persistir objetos e consultar o banco de dados. Entre os parâmetros que o \textit{LocalContainerEntityManagerFactoryBean} recebe estão um adaptador para o uso do \textit{Hibernate}, o nome do pacote onde são criadas entidades e o \classe{ComboPooledDatasource} configurado anteriormente para ser usado como fonte de dados. Ele também recebe um objeto do tipo \textit{Properties} contendo várias configurações do \textit{Java Persistence API} (\textit{JPA}) (linhas 88 e 100). Para informações mais detalhadas sobre o \textit{JPA}, recomenda-se o livro \textit{Pro JPA 2.0}, 2ª edição, da editora \textit{Appress} \cite{19}.

O último método (linha 91) retorna um objeto \classe{JpaTansactionManager}. Este objeto é o responsável por gerenciar as transações com o banco de dados. Ele recebe a instancia do objeto \classe{LocalContainerEntityManagerFactoryBean} configurado anteriormente como parâmetro.

\section{Criando um projeto \anmvc no \textit{Visual Studio Community 2013}}

O link \textit{"new project..."}, na tela inicial do Visual \textit{Studio Community 2013}, mostra um menu com diversos modelos para criação de projetos. Para o projeto desse trabalho, é usando o \textit{"ASP.NET Web Application"} n o menu \textit{"Visual C\#"}.

Um projeto no \textit{Visual Studio} faz parte de uma solução. Uma solução além de ser uma coleção de projetos, contém informações de dependências e configurações. A Figura~\ref{fig:17} ilustra a criação de um novo projeto. Os arquivos de solução junto com os de projeto são os equivalentes aos arquivos de projeto do \textit{Eclipse} e o arquivo \arquivo{build.gradle}.

\figura{vsnewproject.png}{Criando um novo projeto no \textit{Visual Studio Community 2013}}{fig:17}{0.7}

O nome da solução de exemplo é \textit{"NetWebCrud"}. O \textit{Visual Studio} automaticamente dá o mesmo nome da solução para o primeiro projeto criado. O local de armazenamento dos arquivos também pode ser configurado nessa tela. Clicando no botão "OK", aparecerá o menu mostrado na Figura~\ref{fig:18}.

\figura{vsnewproject2.png}{Opções de um novo projeto \textit{web} no \textit{Visual Studio}}{fig:18}{0.7}
{1}
Existem várias opções de projetos para \textit{web}. Para um projeto \textit{MVC} contendo apenas o mínimo necessário para seu funcionamento, é escolhido o modelo \textit{"Empty"} e é marcada a opção \textit{MVC}. A \textit{Microsoft} possui um serviço de hospedagem de aplicações, sites e máqinas virtuais na núvem chamdo de \textit{Microsoft Azure}. O desenvolvedor pode configurar o \textit{Visual Studio} para publicar sua aplicação automaticamente no \textit{Microsoft Azure} utilizando a opção \textit{"Host in the Cloud"}. Como os exemplos deste trabalho não utilizam o \textit{Microsoft Azure}, essa opção foi deixada desmarcada. Clicando em "OK", o \textit{Visual Studio} irá gerar um projeto com a estrutura apresentada pela Figura~\ref{fig:19}.

\figura{19.png}{Estrutura de um projeto \textit{ASP.NET MVC 5}}{fig:19}{0.7}

O \anmvc funciona seguindo o princípio de convenção à configuração. O desenvolvedor deve seguir várias convenções para o desenvolvimento de sua aplicação, em compensação, a quantidade de configuração necessária é mínima.

Todas as páginas e subpastas que contém páginas devem ficar inseridas na pasta \textit{Views}. Todos os \textit{controllers} devem ficar na pasta \textit{Controllers}. No \textit{ASP.NET MVC} existe o conceito de \textit{Areas}, onde o desenvolvedor pode organizar melhor seu código, mas esse conceito não é abordado nesse trabalho. A pasta \textit{Models} pode conter entidades de negócio, repositórios e quaisquer outras classes da camada \textit{Model} do padrão \textit{MVC}, mas o desenvolvedor também pode criar tais classes em outros projetos dentro da solução.

A classe \classe{AssembyInfo} contém informações como nome da aplicação, empresa desenvolvedora e número de versão. Esse arquivo pode ser editado pelo desenvolvedor para atualizar os metadados da aplicação. A pasta \textit{References} contém as bibliotecas que são usadas no projeto. A pasta \textit{App\_Data} pode conter um arquivo de banco de dados minimalista para armazenar dados, mas como é usado o banco de dados \textit{MySql}, essa pasta é apagada manualmente.

A pasta \textit{App\_Start} existe para que o desenvolvedor possa armazenar arquivos de inicialização da aplicação. Um modelo de um novo projeto \anmvc vem com a classe \classe{RouteConfig}, na qual a tabela de rotas que mapeia \textit{URLs} para \textit{controllers} é configurada. O conteúdo desse arquivo é explicado no próximo capítulo durante a criação de \textit{controllers}.

O arquivo \arquivo{Global.asax} contém o método \metodo{Application\_Start}. Esse método é chamado uma vez quando a aplicação é iniciada e nele podem ser incluídas linhas de código que serão executadas quando a aplicação iniciar. No modelo de projeto escolhido, ele já vem com uma chamada ao método \metodo{RegisterRoutes} da classe \classe{RouteConfig} e \metodo{RegisterAllAreas} da classe \classe{AreaRegstration}.

O Arquivo \arquivo{Web.config} na pasta raiz do projeto é um arquivo \textit{XML} que contém diversas configurações para o projeto. O arquivo \arquivo{web.config} na pasta \textit{Views} contém configuração apenas para a criação de \textit{views}.

\subsection{Adicionando dependências usando o \textit{NuGet}}

Do mesmo modo que o \textit{Gradle} pode adicionar bibliotecas externas ao projeto \textit{Java}, o \textit{NuGet} é uma ferramenta que adiciona e gerencia pacotes nos projetos do Visual Studio. Nesse projeto de exemplo são usadas as seguintes bibliotecas:

\begin{itemize}
  \item \lib{MySql.Data.Entity 6.9.5} - Biblioteca necessária para o \textit{MySql} funcionar com o \textit{Entity Framework 6}. Adicionando essa biblioteca, o \textit{NuGet} também adiciona o \textit{driver do MySql} e o \textit{Entity Framework 6} ao projeto.
  \item \lib{Ninject MVC5 3.2.2} - Injeção de dependências para projetos \textit{ASP.NET MVC 5}. 
\end{itemize}

O console do \textit{NuGet} pode ser acessado no menu \textit{TOOLS/NuGet Package Manager/Package Manager Console}, como mostra a Figura~\ref{fig:20}.

\figura{20.png}{Acessando o console do \textit{NuGet}}{fig:20}{0.7}

O console do \textit{NuGet} é exibido no \textit{Visual Studio} como mostra a Figura~\ref{fig:21}. Nele o desenvolvedor pode digitar comandos para procurar e adquirir bibliotecas para seus projetos.

\figura{21.png}{Console do \textit{NuGet}}{fig:21}{0.6}

Para instalar as bibliotecas citadas anteriormente, executa-se os seguintes comandos no console do \textit{NuGet}:

\begin{itemize}
  \item \textit{Install-package mysql.data.entity}
  \item \textit{Install-package ninject.mvc5}
\end{itemize}

Quando o \textit{NuGet} terminar de baixar todos os arquivos, eles são adicionado ao projeto. A Figura~\ref{fig:22} ilustra a instalação das bibliotecas.

\figura{22.png}{Instalação de dependências bem sucedidas}{fig:22}{0.6}

Ao terminar a instalação, os arquivos \arquivo{packages.config} e \arquivo{Web.config} são atualizados com as referências e configurações das novas bibliotecas.

A instalação do \textit{Ninject Mvc 5} adiciona o arquivo \arquivo{NinJectWebCommon.cs} à pasta \textit{App\_Start}. Na Figura~\ref{fig:23} pode-se observar o arquivo adicional citado. O conteúdo e funcionalidade desse arquivo é abordado no capítulo 6.

\figura{23.png}{Arquivo adicional do \textit{Ninject}}{fig:23}{1}

\subsection{Adicionando informações de conexão com o banco de dados}

O acesso ao banco de dados é configurado no arquivo \arquivo{Web.config} na raiz do projeto. Para adicionar informações de conexão com o banco \textit{MySql}, é adicionada a \textit{tag} \textit{connectionStrings} (linha 7) como no exemplo mostrado no código fonte~\ref{lst:6}.

\xmlcode{code/6.txt}{Adicionando configurações de conexão ao \arquivo{Web.xml}}{lst:6}

A \textit{tag} \textit{configuration} é a raiz do arquivo de configuração (linha 2). A \textit{tag} \textit{configSections} (linha 3) deve ser a primeira filha da \textit{tag} raiz, caso não seja, a aplicação irá reportar um erro de configuração. Note que já existem referências ao \textit{Entity Framework} (linha 4). A \textit{tag} \textit{connectionStrings} deve ser filha da \textit{tag} \textit{configuration}.

A \textit{tag} \textit{add} (linha 8) é usada para adicionar informações de conexão com o banco de dados. É necessário dar um nome para a conexão, configurar a \textit{connection string} propriamente dita e o nome da classe responsável pelo acesso ao banco. A \textit{connection string} deve conter no mínimo informações sobre para qual servidor apontar, qual o nome do banco de dados e usuário/senha utilizados. 

\section{Conclusão}

Criar um novo projeto com o \textit{Spring MVC} requer escrita de código, mais passos e conhecimento de como o \textit{framework} funciona internamente. A vantagem de um projeto com \textit{Spring} é a questão da modularidade de componentes (o desenvolvedor adiciona ao projeto somente aquilo que precisa) e uma liberdade maior de configuração.

Uma maneira de tornar a configuração inicial de projetos \textit{Java} com \textit{Spring} mais simples é usar o módulo \textit{Spring Boot}. O \textit{Spring Boot} configura automaticamente a aplicação a partir dos arquivos \textit{.jar} contidos no projeto. Ele detecta as bibliotecas populares e "adivinha" como o projeto deve ser configurado. Internamente ele cria de forma automática a configuração que foi demonstrada na seção 4.1.3.

As vantagens de criação de um novo projeto \textit{ASP.NET MVC} no \textit{Visual Studio} são os modelos de projetos que já vem prontos logo após a instalação. Não existe a necessidade de realizar tantas configurações iniciais, contudo menos aspectos do projeto são configuráveis. O desenvolvedor deve obedecer as convenções do \textit{ASP.NET MVC}.

No próximo capitulo será mostrado como criar \textit{controllers} e \textit{views} nos dois tipos de projetos.

As tabelas~\ref{tab:tbl4} e ~\ref{tab:tbl5} mostram as vantagens e desvantagens de cada tecnologia em relação a criação de projetos.

\begin{table}[h!]   
    \centering
    \Caption{\label{tab:tbl4} Criação de projetos - Vantagens e desvantagens do ambiente Java.}
    \UECEtab{}{
        \begin{tabular}{| L{7cm} | L{7cm}|}
            \toprule
                Vantagens & Desvantagens \\
            \midrule \midrule
                Maior flexibilidade na escolha de bibliotecas, \textit{frameworks} e ferramentas. & Quantidade de configuração inicial consideravelmente maior que a do ambiente \textit{.NET}. \\
            \bottomrule
        \end{tabular}
    }{
        \Fonte{Elaborado pelo autor}
    }
\end{table}

\begin{table}[h!]   
    \centering
    \Caption{\label{tab:tbl5} Criação de projetos - Vantagens e desvantagens do ambiente .NET.}
    \UECEtab{}{
        \begin{tabular}{| L{7cm} | L{7cm}|}
            \toprule
                Vantagens & Desvantagens \\
            \midrule \midrule
                Configuração inicial mínima. & Menor liberdade na escolha de \textit{frameworks} e ferramentas. \\
				Modelos de projetos já prontos. & \\
            \bottomrule
        \end{tabular}
    }{
        \Fonte{Elaborado pelo autor}
    }
\end{table}
