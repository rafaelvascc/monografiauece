\chapter{Fundamentação Teórica}

\section{Desenvolvimento Web}

Desenvolvimento para \textit{web} é o termo utilizado para o desenvolvimento de páginas e/ou sistemas para \textit{internet} ou \textit{intranets}. 
O desenvolvedor cria páginas que são armazenadas em servidores (computadores que servem as páginas para os usuários) e quando requisitadas, são lidas e exibidas em navegadores \textit{web}, como o \textit{Internet Explorer}.

Quando o usuário digita o endereço de uma página no navegador ou clica em um \textit{link} que leva à uma página, uma requisição é enviada ao servidor onde está hospedanda a página. 
O servidor então responde à requisição, enviando a página para o navegador do usuário, sendo exibida na tela do seu computador. A Figura~\ref{fig:requisicao} mostra o fluxo simplificado de uma requisição. 

\figura{requisicao.jpg}{Exemplo de requisição para o servidor}{fig:requisicao}

Páginas da \textit{web} são escritas usando a linguagem HTML (\textit{Hipertext Markup Lanaguage}). 
Essa linguagem define quais elementos devem aparecer no navegador do usuário, como uma caixa de texto ou um botão. 
A aparência desses elementos é definida pela linguagem CSS (\textit{Cascading Style Sheet}).
As versão mais atuais do HTML e do CSS são o HTML 5 e o CSS 3. A Figura~\ref{fig:htmlcss} mostra exemplos dessas linguagens.

\figura{htmlcss.jpg}{Exemplo código HTML e CSS}{fig:htmlcss}

Comportamentos podem ser adicionados à elementos da página, e até para a própria página, utilizando uma linguagem de \textit{script} que o navegador consiga interpretar.
Uma linguagem de \textit{script} utilizada é o \textit{Javascript}. A Figura~\ref{fig:javascript} mostra uma função \textit{Javascript} que é executada quando se pressiona um botão.
Esse é um exemplo simples, código \textit{Javascript} tem a capacidade de manipular uma página e seus elementos de várias maneiras diferentes e também permitir comunicação com o servidor.

\figura{javascript.jpg}{Exemplo código Javascript}{fig:javascript}

Por fim, páginas da \textit{web} podem ser estáticas ou dinâmicas. 
Páginas estáticas são arquivos HTML escritos pelo desenvolvedor \textit{web} de forma que seu conteúdo não é alterado pelo servidor ou pelo usuário.
Páginas dinâmicas são geradas pelo servidor, tendo seu conteúdo modificado pelos dados recebidos de um banco de dados ou pelo \textit{input} do usuário.

\section{O padrão \textit{Model-View-Controller} (MVC)}

O padrão Model-View-Controller foi primeiro documentado em 1979, pelo laboratório \textit{Palo Alto Research Center} da empresa \textit{Xerox}.
O padrão consiste em dividir a aplicação em 3 partes:

\begin{itemize}
  \item \textit{Model} - Representa as entidades e regras de negócio da aplicação.
  \item \textit{View} - A forma como os dados serão exibidos para o usuário.
  \item \textit{Controllers} - A ponte entre \textit{View} e o \textit{Model}. Responde à comandos do usuário e devolve uma resposta.
\end{itemize}

Em uma aplicação \textit{web}, a \textit{View} é uma página exibida pelo navegador \textit{web}. 
Páginas \textit{web} podem possuir botões, \textit{links} e outros elementos que permitam que o usuário interaja com a página e envie informações para o servidor e
o servidor então recebe essas informações (ou comandos) e os trata em classes que fazem o papel de \textit{Controllers}.
Essas classes, em posse dos dados enviados pelo do usuário, decidem que o que fazer com esse \textit{input}. 
Elas podem perceber que o usuário quer apenas ir para a próxima página e envia-la para o seu navegador ou podem enviar esses dados para o \textit{Model}, trata-los de acordo com as regras de negócio da aplicação e devolver o resultado para o usuário.
O resultado pode ser um simples mensagem de sucesso ou falha das operações realizadas na camada de \textit{Model}, ou pode ser uma página dinâmica gerada pelo servidor.

A Figura~\ref{fig:mvc} mostra o fluxo do padrão MVC:

\figura{mvc.jpg}{O fluxo do padrão MVC}{fig:mvc}

\section{\textit{Frameworks}}

\textit{Framework} é definido como um conjunto de classes que cooperam entre si com o objectivo de criar um design reusável para um tipo de \textit{software} especifico \cite{28}.
O \textit{framework} define a arquitetura e estrutura da aplicação. Ele promove uma inversão de controle entre a aplicação e o software na qual ela tem como base. 
Em outras palavras, o \textit{framework} é um conjunto de classes que ajudam o desenvolvedor à escrever sua aplicação. 
O desenvolvedor delega detalhes da arquitetura da aplicação ao \textit{framework} enquanto pode ter um foco maior em desenvolver o negócio.

\subsection{Tipos de \textit{Frameworks} para desenvolvimento \textit{web}}

Quando se trata de desenvolvimento \textit{web}, \textit{frameworks} podem ser de dois tipos: baseados em ações e baseados em componentes.

\textit{Frameworks} baseados em componentes são conhecidos por facilitarem o desenvolvimento \textit{web} para programadores não familiarizados com o desenvolvimento \textit{web}. 
Neles, elementos HTML e código Javascript são gerados pelo \textit{framework}, e esse componentes tem seu comportamento definido no lado servidor da aplicação.
Quando um cliente faz uma requisição à uma aplicação \textit{web} que usa esse tipo de \textit{framework}, a requisição segue o fluxo mostrado pela Figura~\ref{fig:lixo}.

\figura{lixo.jpg}{Fluxo de \textit{framework} baseado em componentes}{fig:lixo}

O ASP.NET WebForms e o JSF (Java Server Faces) são exemplos de \textit{frameworks} baseados em componentes. O uso dese tipo de framework não é recomendado para aplicações \textit{web} modernas, 
pois a Microsoft descontinuou o ASP.NET WebForms e a empresa de consultoria ThoughtWorks (na qual Martin Fowler é cientista chefe) recomenda evitar esse tipo de \textit{framework}, 
argumentando que suas equipes tiveram vários problemas ao usar o JSF.

\textit{Frameworks} baseados em ações tratam requisições no \textit{Controller} e não na \textit{View}. 
Eles exigem um maior conhecimento do desenvolvedor de como se programar para a \textit{web}, gerando menos componentes HTML e não gerando Javascript automaticamente. 
As requisições nesse tipo de \textit{framework} são apssados do \textit{Controller} para o \textit{Model}, o \textit{Controller} então passa a resposta do \textit{Model} para a \textit{View} como mostra a Figura~\ref{fig:lixo}.

\figura{certo.jpg}{Fluxo de \textit{framework} baseado em ações}{fig:certo}

Os \textit{frameworks} \textit{Spring MVC} e \textit{ASP.NET MVC} são baseados em ações.
