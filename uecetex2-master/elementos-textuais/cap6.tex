\chapter{Entidades e repositórios}

Entidades são classes que representam os objetos de negócio. A classe \textit{Person} utilizada até agora é um exemplo de entidade. Em ambos os projetos, essa classe é usada para gerar uma tabela no banco de dados e substituir as listas em memória que foram usadas até então. Essa técnica é chamada de mapeamento objeto/relacional, pois mapeia propriedades de objetos para tabelas e colunas no banco de dados relacional \cite{13}. Nos exemplos desse capítulo são usados o \textit{Hibernate} no projeto \textit{Java/Spring} e o \textit{Entity Framework} no projeto \textit{ASP.NET MVC} como bibliotecas para mapeamento objeto/relacional.

Além da classe \textit{Person}, nesse capítulo é criada a classe \textit{Contact} que conterá o \textit{e-mail} e o telefone de uma pessoa. Essa classe terá um relacionamento de um para um com a classe \textit{Person} e irá gerar uma chave estrangeira no banco de dados.

Com as entidades prontas, são feitos repositórios para persistí-las no banco de dados. Será feito um repositório genérico e um repositório específico para cada classe.

\section{Entidades}

As seções a seguir mostram como utilizar uma classe para gerar tabelas no banco de dados. Entidades precisam de um identificador único para ser sua chave primária, essa propriedade será chamada \textit{Id} e será do tipo inteiro.

Como são usadas duas entidades nesse capítulo e ambas precisam do identificador único, é criada uma classe base com essa propriedade em comum chamada \textit{BaseEntity}. O intuito da criação dessa classe é mostrar como é possível o uso de herança nas entidades. As classes \textit{Person} e \textit{Contact} são especializações de \textit{BaseEntity}.

\subsection{\textit{Java}}

Entidades no projeto \textit{Java/Spring} são criadas no pacote \textit{br.uece.webCrud.model}. A classe \textit{BaseEntity} é criada como mostra o código fonte~\ref{lst:32}.

\javacode{code/32.txt}{Classe \textit{BaseEntity} no projeto \textit{Java}}{lst:32}

A classe \textit{BaseEntity} é decorada com a anotação \textit{@MappedSuperclass} (linha 6) que diz ao \textit{Hibernate} que entidades que sejam suas subclasses devem levar em consideração as propriedades herdadas no mapeamento objeto/relacional. A propriedade \textit{id} é decorada com as anotações \textit{@Id}, que configura o \textit{Hibernate} para usá-la como chave primária e única, e \textit{@GeneratedValue}, que delega ao banco de dados a lógica de geração do valor da chave (linhas 9 à 11).

A classe \textit{Person} então é modificada como no código fonte~\ref{lst:33}.

\javacode{code/33.txt}{Classe \textit{Person} herda de \textit{BaseEntity}}{lst:33}

A classe \textit{Person} agora é subclasse de \textit{BaseEntity} e está decorada com a anotação \textit{@Entity} (linhas 9 e 10). A anotação \textit{@Entity} diz ao \textit{Hibernate} que essa classe deve ser mapeada para uma tabela e as propriedades da classe para as colunas da tabela.

Por padrão, o \textit{Hibernate} procura por uma tabela com o mesmo nome da classe e colunas com o mesmo nome das propriedades para realizar o mapeamento, mas esse comportamento pode ser modificado com as anotações \textit{@Table}, \textit{@Column}, \textit{@Transient} e muitas outras. A propriedade \textit{name} por exemplo, está decorada com a anotação \textit{@Column} (linha 11) especificando que aquela propriedade não pode ser nula no banco de dados e a propriedade \textit{birthDate} está decorada com a anotação \textit{@Temporal} (linha 14) especificando que a coluna no banco deve ser do tipo \textit{Date}.

Na seção 4.1.3, durante a configuração do projeto, a propriedade \textit{hibernate.hbm2ddl.auto} foi configurada com o valor \textit{"create"} (código fonte 5, linhas 32 e 105), assim o \textit{Hibernate} cria as tabelas no banco de dados de acordo com as configurações de classes decoradas com a anotação \textit{@Entity}. Outros valores para essa configuração podem ser \textit{"update"}, para atualizar a estrutura das tabelas sem perder dados, e \textit{"validate"}, para apenas verificar se o mapeamento objeto/relacional está consistente. O \textit{Hibernate} cria tabelas, mas não cria o banco de dados, para que a criação de tabelas funcione é necessário que o banco de dados já exista. As tabelas são criadas no momento em que a aplicação é iniciada.

\subsection{\textit{.NET}}

Entidades no projeto ASP.NET MVC são criadas na pasta \textit{Model/Entities}. Considere que a classe \textit{Person} é movida para essa pasta. O código fonte~\ref{lst:34} mostra a classe \textit{BaseEntity} do projeto \textit{ASP.NET MVC}.

\javacode{code/34.txt}{Classe \textit{BaseEntity} no projeto \textit{ASP.NET}}{lst:34}

Uma convenção do \textit{Entity Framework} é que se uma classe tiver uma propriedade chamada \textit{Id} de tipo numérico ou \textit{guid} (tipo de identificar único), essa propriedade será usada como chave primária. Assim nenhuma configuração adicional é necessária.

Assim como no projeto \textit{Java}, a classe \textit{Person} do projeto \textit{ASP.NET MVC} é modificada para ser subclasse de \textit{BaseEntity}, como mostrado no código fonte~\ref{lst:35}.

\javacode{code/35.txt}{Classe \textit{Person} no projeto \textit{ASP.NET} agora herda de \textit{BaseEntity}}{lst:35}

Assim como no projeto \textit{Java}, entidades do \textit{Entity Framework} podem ser decoradas com anotações que controlam como as tabelas serão geradas. No exemplo anterior a anotação \textit{Required} também diz que a coluna correspondente à propriedade \textit{Name} não pode aceitar valores nulos (linha 7).

\subsubsection{Criando o contexto do banco de dados}

Não existe configuração nas classes do projeto \textit{ASP.NET MVC} que indique que elas são entidades e que devem ser usadas no mapeamento objeto/relacional. Esse papel é da classe \textit{DbContext} do \textit{Entity Framework}. A classe \textit{DbContext} representa o banco de dados como um todo. Para criar o banco de dados a partir das entidades, o desenvolvedor deve criar uma classe que herde de \textit{DbContext} e tenha propriedades do tipo \textit{DbSet} como a classe \textit{NetWebCrudContext} no exemplo do código fonte~\ref{lst:36}. 

\sharpcode{code/36.txt}{Classe \textit{NetWebCrudContext} representa o contexto do banco de dados}{lst:36}

A classe \textit{NetWebCrudContext} possui um \textit{DbSet} de objetos do tipo \textit{Person} (linha 19) que representa a tabela no banco de dados. Para utilizar o banco, o desenvolvedor deve criar uma instância do contexto e fazer consultas aos \textit{DbSets} usando o Language Integrated Query (\textit{LINQ}). O \textit{LINQ} é uma linguagem para realizar consultas em coleções do \textit{.NET Framework}. Usado em \textit{DbSets}, ele gera consultas \textit{SQL} automaticamente para manipular o banco de dados. Consultas com \textit{LINQ} são abordadas na seção 7.4.

O construtor de \textit{NetWebCrudContext} executa o construtor de \textit{DbContext} recebendo como parâmetro o nome da \textit{connection string} que contém informações de conexão com o banco (linha 13). Ele recebe o nome da \textit{connection string} \textit{NetWebCrudContext}, a mesma que foi configurada no arquivo \textit{Web.xml} na seção 4.2.2.

O construtor do contexto do banco de dados também é usado para configurar o tipo de inicialização do banco. O construtor de \textit{NetWebCrudContext} utiliza como inicializador a classe \textit{DropCreateDatabaseIfModelChanges} (linhas 15 e 16), então se qualquer entidade for modificada, o banco de dados é totalmente reconstruído e todos os dados perdidos. Existem outras classes que podem ser usadas como inicializadores como \textit{CreateDatabaseIfNotExists}, para criar o banco apenas uma vez, e \textit{DropCreateDatabaseAways} para sempre recriar o banco de dados quando a aplicação iniciar. O desenvolvedor pode estender esses inicializadores ou construir um completamente novo que implemente a interface \textit{IDatabaseInitializer}.

O \textit{Entity Framework} não atualiza automaticamente a estrutura do banco de dados sem perder informações, para isso é necessário utilizar o \textit{Entity Framework Migrations}. Essa funcionalidade permite utilizar as linguagens \textit{C\#} ou \textit{VB.NET} para executar \textit{scripts} no banco de dados e atualizar sua estrutura. O \textit{Entity Framework Migrations} não é abordado nesse trabalho, para mais informações recomenda-se consultar o livro \textit{Code-First Development with Entity Framework} da editora \textit{Packt Publishing} \cite{27}.

O método \metodo{OnModelCreating} (linha 21) é usado para configurar diversos aspectos da criação do banco de dados.  No exemplo da linha 23 do quadro~\ref{lst:36}, a convenção de pluralizar o nome das tabelas é removida.

Diferente do \textit{Hibernate} que cria ou atualiza o banco de dados quando a aplicação é iniciada, o \textit{Entity Framework} cria o banco de dados na primeira vez em que uma instância de \textit{DbContext} é criada.

Uma instância de \textit{NetWebCrudContext} precisa ser construída para que se possa usar o banco de dados. No projeto exemplo, cada requisição de um usuário deve criar uma nova instância de \textit{NetWebCrudContext}. Se mais de um usuário utilizar a mesma instância, poderão haver erros de concorrência e um usuário poderá ter acesso os dados de outro que ainda não foram persistidos no banco de dados. É por esse motivo que o método \textit{InRequestScope} é utilizado nos exemplos de injeção de dependências do projeto \textit{.NET}. O \textit{Ninject} é responsável por criar as instâncias de \textit{NetWebCrudContext} e injetá-las nos repositórios que serão criados na seção 7.3.2. O código fonte~\ref{lst:37} mostra a configuração de injeção de \textit{NetWebCrudContext}.

\sharpcode{code/37.txt}{Injeção do contexto do banco de dados}{lst:37}

\section{Relacionamentos entre entidades}

Entidades podem se relacionar com outras com o uso de chaves estrangeiras. O relacionamento entre a classe \textit{Person} e a classe \textit{Contact} é usado nos exemplos dessa seção.

\subsection{\textit{Java}}

A classe \textit{Contact} é criada no pacote \textit{br.uece.webCrud.model} como mostra o código fonte~\ref{lst:38}.

\javacode{code/38.txt}{Classe \textit{Contact} no projeto \textit{Java}}{lst:38}

Assim como a classe \textit{Person}, a classe \textit{Contact} herda de \textit{BaseEntity} e é decorada com a anotação \textit{@Entity} (linhas 5 e 6). O Relacionamento com a classe \textit{Person} é configurado pela anotação \textit{@OneToOne} decorando o atributo \textit{person} (linhas 11 e 12). O atributo \textit{mappedBy} recebendo o valor \textit{"contact"} quer dizer que na classe \textit{Person} deve existir um atributo com esse nome, configurado como o dono do relacionamento entre as tabelas. Resumindo, a tabela correspondente à classe \textit{Person} deve ter a coluna com a chave estrangeira para \textit{Contact}.

Além da anotação \textit{@OneToOne}, o \textit{Java Persistence API} possui as anotações \textit{@OneToMany}, \textit{@ManyToOne} e \textit{@ManyToMany} para criar relacionamentos. As anotações \textit{@OneToMany} e \textit{@ManyToMany} devem ser usadas em uma coleção de objetos, quando uma entidade tem um relacionamento de "um para muitos" ou de "muitos para muitos" com outra entidade.

O relacionamento na classe \textit{Person} também deve ser configurado, como mostra o código fonte~\ref{lst:39}.

\javacode{code/39.txt}{Classe \textit{Person} com relacionamento para \textit{Contact}}{lst:39}

A classe \textit{Person} agora possui uma propriedade do tipo \textit{Contact}, também decorada com a anotação \textit{@OneToOne} e com a anotação \textit{@JoinColumn} (linhas 22 à 24). A primeira anotação é configurada para cascatear todas as operações de \textit{Person} para \textit{Contact}. Se, por exemplo, um objeto do tipo \textit{Person} for atualizado no banco de dados, a sua propriedade \textit{contact} também será atualizada. A propriedade \textit{fetch} recebe o valor \textit{FetchType.EAGER}, configurando a propriedade \textit{contact} para ser carregada do banco de dados por \textit{eager loading} (executando uma única consulta, ao mesmo tempo quando um objeto da classe \textit{Person} é recuperado do banco). A propriedade \textit{fetch} também pode receber \textit{FetchType.LAZY}, fazendo com que \textit{contact} só seja carregado quando necessário, mas gerando uma nova consulta ao banco de dados.

A anotação \textit{@JoinColumn} configura a classe \textit{Person} para criar a coluna da chave estrangeira na sua tabela correspondente. Essa coluna tem o nome de \textit{contactId}. O \textit{Hibernate} detecta automaticamente as propriedades decoradas com \textit{@Id} e as utiliza como chaves estrangeiras em tabelas relacionadas. A Figura~\ref{fig:36} mostra o resultado de criação das tabelas no banco de dados \textit{MySQL}. 

\figura{36.png}{Tabelas geradas pelo \textit{Hibernate}}{fig:36}{1}

\subsection{\textit{.NET}}

A classe \textit{Contact}, ilustrada no código fonte~\ref{lst:40}, também é criada no projeto \textit{ASP.NET MVC}.

\sharpcode{code/40.txt}{Classe \textit{Contact} no projeto \textit{ASP.NET MVC}}{lst:40}

A classe \textit{Contact} não requer nenhuma configuração, sendo apenas preciso herdar de \textit{BaseEntity} para adquirir a propriedade \textit{Id}. Por outro lado, a classe \textit{Person} recebe mais propriedades como mostra o código fonte~\ref{lst:41}.

\sharpcode{code/41.txt}{Classe \textit{Person} no projeto \textit{ASP.NET} com novas propriedades}{lst:41}

Na classe \textit{Person} é adicionada uma propriedade \textit{Contact} modificada com a palavra chave \textit{"virtual"} para configurá-la como uma propriedade de navegação que pode ser carregada por \textit{lazy loading} (linha 10). Isso resulta em mais uma consulta no banco, mas com o uso do método \textit{Include} do \textit{LINQ} é possível recuperar as informações de \textit{Person} e \textit{Contact} com apenas uma consulta.

A propriedade \textit{ContactId} representa a coluna da chave estrangeira (linha 8). Por convenção do \textit{Entity Framework}, propriedades de chaves estrangeiras devem ter o nome da propriedade de navegação (\textit{Contact}) seguido do nome da sua chave primária (\textit{Id}).

Por último, é necessário adicionar um \textit{DbSet} para \textit{Contact} no contexto do banco de dados, como mostra a linha 20 do código fonte~\ref{lst:42}.

\sharpcode{code/42.txt}{Classe \textit{NetWebCrudContext} com o \textit{DbSet} para \textit{Contact}}{lst:42}

O resultado no banco é semelhante ao do projeto \textit{Java/Spring}, como mostra a Figura~\ref{fig:37}. Uma diferença que deve ser ressaltada é a criação da tabela \textit{\_\_migrationhistory}. Essa tabela é usada pelo \textit{Entity Framework} para armazenar dados sobre o histórico da estrutura do banco de dados. Assim o \textit{Entity Framework Migrations} sabe que alterações devem ser feitas na estrutura do banco quando as entidades são modificadas.

\figura{37.png}{Tabelas geradas pelo \textit{Entity Framework}}{fig:37}{1}

\section{Repositórios}

A persistência e acesso aos dados são feitos utilizando o padrão de repositórios. Em cada projeto é criado um repositório genérico, que pode realizar operações básicas (salvar, atualizar, recuperar e apagar) em qualquer classe que seja subclasse de \textit{BaseEntity}. Um repositório especifico para a entidade \textit{Person} é criado nos dois projetos e outro para \textit{Contact} apenas no projeto \textit{ASP.NET MVC}. Eles são especializações do repositório genérico e possuem operações especificas para essas entidades.

É demonstrado como utilizar a \textit{Java Persistence Query Language} (\textit{JPQL}) e a \textit{Language Integrated Query} (\textit{LINQ}) para realizar consultas ao banco de dados e como gerenciar transações. Por último, os repositórios são injetados nas classes de serviço, substituindo as listas em memória.

\subsection{\textit{Java}}

Nos exemplos dessa seção, os repositórios e interfaces que eles implementam no projeto \textit{Java/Spring} são criados no pacote \textit{br.uece.webCrud.repositories}.

Primeiramente, são criadas a interface \textit{GenericRepository} e a classe \textit{GenericRepositoryImpl}, demonstradas nos codigos fonte~\ref{lst:43} e~\ref{lst:44}, respectivamente. Essa classe é o repositório genérico mencionado anteriormente.

\javacode{code/43.txt}{\textit{Interface} \textit{GenericRepository}}{lst:43}

\javacode{code/44.txt}{Classe \textit{GenericRepositoryImpl}}{lst:44}

A classe \textit{GenericRepositoryImpl} é decorada com a anotação \textit{@Repository}, então instâncias dela podem ser injetadas pelo \textit{Spring IoC Container}. Erros de seus métodos em tempo de execução, dispararam a exceção \textit{DataAccessException} do \textit{Spring Framework}.

Essa classe faz uso de \textit{Java Generics}, podendo realizar suas operações com qualquer classe que herde de \textit{BaseEntity}. Seu construtor extrai o tipo da classe que está sendo utilizada em suas operações (linhas 19 e 20). \textit{GenericRepositoryImpl} é uma classe abstrata (que não pode ser instanciada), suas subclasses é que irão efetivamente executar operações.

O objeto do tipo \textit{EntityManager}, decorado com a anotação \textit{@PersistenceContext}, é que executa as operações no banco de dados (linhas 19 e 20). A instância de \textit{EntityManager} é criada pelo \textit{bean} \textit{LocalContainerEntitityManagerFactoryBean}, configurado na seção 4.1.3 (código fonte 5, linhas80 à 90), e injetado pelo \textit{Spring IoC Container}. Para objetos especificos utilizado em acesso à dados, a anotação \textit{@PersistenceContext} funciona do mesmo modo que \textit{@Autowired}.

O método \metodo{getAll} (linha 24) retorna todas as entradas de uma determinada entidade armazenada no banco de dados. Ele mostra um exemplo de como criar uma consulta usando o método \metodo{createQuery} de \textit{EntityManager}, recebendo o texto da consulta escrito em \textit{JPQL}. Essa linguagem possui similaridades com \textit{SQL}. O método \metodo{getResultList} executa a consulta criada, convertendo o texto \textit{JPQL} em \textit{SQL}, realizando a consulta no banco de dados, mapeando tuplas para objetos, e devolvendo o resultado como uma lista. Para mais informações sobre o \textit{JPQL}, é recomendado consultar a documentação da linguagem no site \url{http://docs.oracle.com/cd/E15523_01/apirefs.1111/e13946/ejb3_langref.html}.

O método \metodo{getById} (linha 31) usa o método \metodo{find} do \textit{EntityManager} para encontrar uma entidade com chave primária e tipo que foram passados como parâmetros. O método \metodo{add} (linha 36) chama o método \metodo{persists} para inserir um novo objeto no banco de dados. O objeto à ser persistido não pode ter uma chave primária de valor já existente no banco, ou o \textit{Hibernate} irá disparar uma exceção \textit{PersistenceException}.

O método \metodo{merge}, chamado por \metodo{update} (linha 42), realiza atualização dos valores de uma entidade no banco de dados. Para que ele funcione é necessário que a chave primária do objeto que vai ser atualizado exista, ou o \textit{Hibernate} também disparará uma exceção. O método \metodo{delete}, usa \metodo{getById} para recuperar uma entidade pela sua chave primária e chama o método \metodo{remove} de \textit{EntityManager} para remove-la do banco de dados.

Com o repositório genérico criado, é possível criar um repositório específico para a classe \textit{Person}, como mostram os códigos fonte~\ref{lst:45} e~\ref{lst:46}. A classe \textit{PersonRepositoryImpl} adiciona o método \metodo{existsWithName} (linha 9), que verifica a existência de uma pessoa com um certo nome. Nesse método é criada uma consulta mais complexa, que retorna o valor booleano \textit{true}, se existir uma entidade no banco de dados com a propriedade \textit{name} igual a passada como parâmetro e propriedade \textit{id} diferente da passada como parâmetro. Esse exemplo também mostra como adicionar parâmetros às consultas, utilizando método \metodo{setParameter}.

\javacode{code/45.txt}{Interface PersonRepository}{lst:45}

\javacode{code/46.txt}{Classe PersonRepositoryImpl}{lst:46}

Por último, a lista em memória da classe \textit{PersonServiceImpl} é substituída por \textit{PersonRepository} (linhas 13 e 14) e a anotação \textit{@Transactional} é adicionada à classe (linha 10), como mostra o código fonte~\ref{lst:47}.

\javacode{code/47.txt}{\textit{PersonServiceImpl} utilizando \textit{PersonRepository}}{lst:47}

A anotação \textit{@Transactional} do \textit{Spring Framework} pode ser utilizada em classes ou métodos. Essa anotação configura o gerenciamento de transações com o banco de dados. Decorando um método, essa anotação faz com que uma única transação seja usada em todas as operações de acesso a dados dentro daquele método. Decorando uma classe, como no exemplo do código fonte~\ref{lst:47}, a anotação aplica o seu efeito a todos os métodos da classe.  O \textit{Spring} controla internamente a criação e reutilização de transações já existentes usando o \textit{bean} \textit{JpaTransactionManager} também configurado na seção 4.1.3 (código fonte 5, linha 93). Se um método decorado com essa anotação for chamado internamente por outro também decorado com ela, o método interno irá aproveitar a transação que já existe e não irá criar uma nova. Esse comportamento padrão pode ser modificado configurando atributos da anotação. 

Caso não ocorram erros nas operações com o banco de dados, as operações da transação são persistidas de forma atômica. Caso haja algum erro, as operações são revertidas e uma exceção é disparada. De toda forma, a transação é destruída automaticamente quando não é mais utilizada.

Como o relacionamento de \textit{Person} com \textit{Contact} configurado com \textit{CascadeType.ALL} na seção 7.2.1, quando um objeto do tipo \textit{Person} é salvo ou atualizado no banco de dados, o seu contato também é. Então nos exemplos desse capitulo para \textit{Java}, não houve a necessidade de se criar um repositório para a classe \textit{Contact}.

\subsection{\textit{.NET}}

Os repositórios no projeto \textit{ASP.NET MVC} são criados na pasta \textit{Models/Repositories}. Assim como no projeto \textit{Java/Spring}, começa-se criando o repositório genérico e sua \textit{interface}. O código fonte~\ref{lst:48} mostra a \textit{interface} do repositório genérico, \textit{IGenericRepository}, e o código fonte~\ref{lst:49} mostra sua implementação, \textit{GenericRepository}.

\sharpcode{code/48.txt}{\textit{Interface} \textit{IGenericRepository}}{lst:48}

\sharpcode{code/49.txt}{Classe \textit{GenericRepository}}{lst:49}

Assim como no projeto \textit{Java/Spring}, o repositório genérico é uma classe abstrata que realiza operações no banco de dados para classes que herdam de \textit{BaseEntity}. Também semelhante ao outro projeto, existe uma propriedade no repositório que efetivamente irá persistir dados e consultar o banco, a propriedade \textit{context} do tipo \textit{NetWebCrudContext} (linha 17). A propriedade \textit{EntitySet} é usada para facilitar o acesso ao \textit{DbSet} correto da entidade a qual o repositório genérico irá manipular (linha 19).

O método \metodo{GetById} (linha 24) mostra um exemplo de como se usar expressões \textit{lambda} e a sintaxe fluente do \textit{LINQ} para realizar uma consulta. Métodos de filtragem e ordenação da sintaxe fluente do \textit{LINQ} retornam coleções do tipo \textit{IQueryable}. Essa \textit{interface} permite que mais métodos sejam chamados em cascata para a criação de consultas complexas. Uma consulta do \textit{LINQ} só é realmente executada no banco de dados quando se chamam certos métodos, como \metodo{Single}, \metodo{First}, \metodo{ToList} e \metodo{ToArray}.  Essas consultas também são transformadas em \textit{scripts} \textit{SQL} quando são executadas no banco de dados. No método \metodo{GetAll} (linha 29), o método \metodo{ToList} do \textit{DbSet} retorna todas as tuplas do banco de dados transformadas em uma lista de objetos.

Dentro do método \metodo{Save} (linha 34), um objeto é adicionado ao \textit{DbSet} usando o método \metodo{Add}, equivalente a uma operação de inserção na tabela. Operações de inserção, atualização e deleção de dados só são persistidas após a chamada do método \metodo{SaveChanges} do contexto. Esse método cria uma transação que, em caso de erro, reverte operações e dispara uma exceção. É possível ter um controle manual sobre transações, como ainda será visto nessa seção.

O método \metodo{Update} (linhas 41) é um pouco mais complexo. Para persistir dados atualizados de uma entidade no banco de dados, caso a entidade não esteja anexada ao contexto, é necessário anexa-la e utilizar o contexto para modificar seu estado para \textit{Modfied}. Assim, quando \metodo{SaveChanges} for chamado, o contexto saberá que aquela entidade é de um objeto que já existe no banco de dados (através da chave primária já existente) e irá atualizar suas informações.

O método \metodo{Delete} (linha 49) é bastante similar ao do repositório do projeto \textit{Java/Spring}. Uma entidade é recuperada pela sua chave primária e o método \metodo{Remove} do \textit{DbSet} a remove do banco de dados.

O método \metodo{GetTransaction} (linha 59) proporciona um melhor controle sobre transações. O controle automático de transações que o \textit{Entity Framework} possui no método \metodo{SaveChanges} pode não ser suficiente para conversações extensas com o banco de dados em regras de negócio complexas. Então é possível criar uma transação manualmente, com o método \metodo{BeginTransation}, e utilizar essa transação para persistir ou reverter alterações. Mesmo com o uso dessa transação, e necessário chamar o método \metodo{SaveChanges} antes executar a persistência de dados.

Os repositórios para as entidades \textit{Person} e \textit{Contact} são criados como mostram os códigos fonte~\ref{lst:50} e ~\ref{lst:51}. O \textit{Entity Framework} cascateia automaticamente operações de inserção e deleção entre entidades relacionadas, mas não de atualização. Sendo assim, é necessária a existência de repositório de \textit{Contact} para que atualizações dessa entidade sejam persistidas.

\sharpcode{code/50.txt}{\textit{Interfaces} \textit{IPersonRepository} e \textit{IContactRepository}}{lst:50}

\sharpcode{code/51.txt}{Classes \textit{PersonRepository} e \textit{ContactRepository}}{lst:51}

Assim como no projeto \textit{Java/Spring}, o repositório para a classe \textit{Person} possui um método que verifica a existência de uma outra entidade que possua o mesmo nome. A consulta é feita com uma expressão \textit{lambda} do \textit{LINQ}. O repositório para \textit{Contact} não possui nenhuma operação adicional, usa apenas as operações herdades de \textit{GenericRepository}. Como esses repositórios também serão injetados na classe de serviço \textit{PersonService}, sua configuração de injeção é adicionada à classe \textit{NinjectWebCommon}, como mostra o código fonte~\ref{lst:52} (linhas 4 e 5).

\sharpcode{code/52.txt}{Configuração de injeção de repositórios no projeto \textit{ASP.NET MVC}}{lst:52}

Por último, a classe de serviço \textit{PersonService} é atualizada como no código fonte~\ref{lst:53}. O método \metodo{Update} (linha 33) mostra como fazer controle manual de transações. Chamando o método \metodo{GetTransaction} de \textit{PersonRepository}, ou de qualquer outro repositório, uma transação é criada para o contexto \textit{NetWebCrudContext} que é injetado no escopo da requisição do usuário. Dessa forma, só existe uma instância do contexto do banco de dados sendo utilizada por todos os repositórios, então a transação também é única.

Usando uma transação dessa forma, o desenvolvedor pode fazer as operações que desejar com quaisquer repositórios, e chamar o método \metodo{Commit} (linha 44) da transação para finalizar a persistência. Em caso de erros, é chamado o método \metodo{Rollback} (linha 47) para reverter às alterações no banco de dados. Como a transação está dentro do escopo da palavra chave \textit{using}, o coletor de lixo do \textit{.NET Framework} se encarrega de chamar o método \metodo{Dispose} automaticamente para finalizar a transação.

\sharpcode{code/53.txt}{Classe \textit{PersonService} utilizando ambos os repositórios no projeto \textit{ASP.NET MVC}}{lst:53}

\section{Conclusão}

A criação de entidades para o modelo objeto/relacional com \textit{Hibernate} e \textit{Entity Framework} possuem abordagens distintas. No primeiro, classes precisam de anotações para serem tratadas como entidades, e no segundo não há essa necessidade. No \textit{Entity Framework}, classes mais simples podem ser usadas e convenções cuidam de lógica no mapeamento objeto/relacional. A desvantagem do uso do \textit{Entity Framework} é a necessidade de criação de uma classe de contexto que representa o banco de dados. Dependendo da complexidade da aplicação, a codificação e manutenção da classe de contexto pode se tornar uma inconveniência.

O relacionamento entre entidades no \textit{Hibernate} é feito com anotações na própria entidade, e a configuração de relacionamentos de muito para muitos, por exemplo, pode se tornar complicada de entender para um desenvolvedor iniciante. O \textit{Entity Framework} abstrai essa configuração, sendo apenas necessária a criação de propriedades de navegação entre entidades e propriedades que representem chaves estrangeiras.

Repositórios em projetos \textit{Spring} com \textit{Hibernate} utilizam uma implementação de \textit{EntityManager} para manipular entidades, podendo usar \textit{Criteria Queries} ou \textit{JPQL} para gerar consultas. \textit{Criteria Queries} utilizam vários métodos para gerar uma consulta, que no fim resulta em um código complexo, de baixa legibilidade e de manutenção questionável. A linguagem \textit{JPQL} se assemelha a \textit{SQL} e é mais legível, mas tem a desvantagem de ser escrita com \textit{strings}, o que a torna mais propensa a erros. Em projetos com o \textit{Entity Framework}, a linguagem \textit{LINQ} é usada para gerar consultas. Essa linguagem é de fácil entendimento, e pode utilizar as vantagens do \textit{IntelliSense} (ferramenta de auto completar código do \textit{Visual Studio}), tornando a detecção de erros mais fácil e a escrita de código mais eficiente.

Quanto ao controle de transações, o modo automático como o \textit{Spring Framework} lida com transações, o torna uma melhor opção nesse aspecto do que o \textit{Entity Framework}.

As tabelas~\ref{tab:tbl10} e ~\ref{tab:tbl11} mostram as vantagens e desvantagens de cada tecnologia em relação a criação de repositórios e entidades.

\begin{table}[h!]   
    \centering
    \Caption{\label{tab:tbl10} Criação de repositórios e entidades - Vantagens e desvantagens do ambiente Java.}
    \UECEtab{}{
        \begin{tabular}{| L{7cm} | L{7cm}|}
            \toprule
                Vantagens & Desvantagens \\
            \midrule \midrule
                Linguagem \textit{JPQL} para a criação de \textit{queries} se assemelha ao \textit{SQL}. & Relacionamentos entre entidades e outras de suas características devem ser configuradas com anotações. \\
				Controle de transações automático. & O \textit{JPQL} é escrito com \textit{Strings} comuns, facilitando a ocorrência de erros. \\
				Mudanças no modelo das entidades podem ser refletidas automaticamente no banco de dados, sem perda de dados, pelo \textit{Hibernate}. & Mapeamento de \textit{URLs} para ações espalhados pelos \textit{controllers}.\\
            \bottomrule
        \end{tabular}
    }{
        \Fonte{Elaborado pelo autor}
    }
\end{table}

\begin{table}[h!]   
    \centering
    \Caption{\label{tab:tbl11} Criação de repositórios e entidades - Vantagens e desvantagens do ambiente .NET.}
    \UECEtab{}{
        \begin{tabular}{| L{7cm} | L{7cm}|}
            \toprule
                Vantagens & Desvantagens \\
            \midrule \midrule
                Convenções para a criação de relacionamentos e outras características de entidades. & Necessária a criação do contexto do banco de dados para definir classes como entidades. \\
				Linguagem \textit{LINQ} fortemente tipada para a criação de \textit{queries}. & Mudanças no banco de dados devem ser feitas utilizando o \textit{Entity Framework Migrations}.\\
            \bottomrule
        \end{tabular}
    }{
        \Fonte{Elaborado pelo autor}
    }
\end{table}