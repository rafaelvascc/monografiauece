\chapter{Entidades e repositórios}

Entidades são classes que representam os objetos de negócio, a classe \classe{Person} utilizada até agora é um exemplo de entidade. Em ambos os projetos, essa classe é usada para gerar uma tabela no banco de dados e substituir as listas em memória que foram usadas até então. Essa técnica é chamada de mapeamento objeto/relacional, pois mapeia propriedades de objetos para tabelas e colunas no banco de dados relacional. Nos exemplos desse capitulo são usados o \est{Hibernate} no projeto \est{Java/Spring} e o \est{Entity Framework} no projeto \est{ASP.NET MVC} como bibliotecas para mapeamento objeto/relacional.

Além de a classe \classe{Person}, nesse capitulo é criada a classe \classe{Contact} que conterá o \est{e-mail} e o telefone de uma pessoa. Essa classe terá um relacionamento de um para um com a classe \classe{Person} e irá gerar uma chave estrangeira no banco de dados.

Com as entidades prontas, são feitos repositórios para persisti-las no banco de dados. Será feito um repositório genérico e um repositório especifico para cada classe.

\section{Entidades}

As seções a seguir mostram como utilizar uma classe para gerar tabelas no banco de dados. Entidades precisam de um identificador único para ser sua chave primária, essa propriedade será chamada \est{Id} e será do tipo inteiro.

Como são usadas duas entidades nesse capitulo e ambas precisam do identificador único, é criada uma classe base com essa propriedade em comum chamada \classe{BaseEntity}. O intuito da criação dessa classe é mostrar como é possível o uso de herança nas entidades. As classes \classe{Person} e \classe{Contact} são especializações de \classe{BaseEntity}.

\subsection{\lang{Java}}

Entidades no projeto \est{Java/Spring} são criadas no pacote \pacote{br.uece.webCrud.model}. A classe \classe{BaseEntity} é criada como mostra o quadro~\ref{lst:32}.

\javacode{code/32.txt}{Classe \classe{BaseEntity} no projeto \lang{Java}}{lst:32}

A classe \classe{BaseEntity} é decorada com a anotação \annotation{@MappedSuperclass} que diz ao \est{Hibernate} que entidades que sejam suas subclasses devem levar em consideração as propriedades herdadas no mapeamento objeto/relacional. A propriedade \est{id} é decorada com as anotações \annotation{@Id}, que configura o \est{Hibernate} para usá-la como chave primária e única, e \annotation{@GeneratedValue}, que delega ao banco de dados a lógica de geração do valor da chave.

A classe \classe{Person} então é modificada como no quadro~\ref{lst:33}.

\javacode{code/33.txt}{Classe \classe{Person} herda de \classe{BaseEntity}}{lst:33}

A classe \classe{Person} agora é subclasse de \classe{BaseEntity} e está decorada com a anotação \annotation{@Entity}. A anotação \annotation{@Entity} diz ao \est{Hibernate} que essa classe deve ser mapeada para uma tabela e as propriedades da classe para as colunas da tabela.

Por padrão o \est{Hibernate} procura por uma tabela com o mesmo nome da classe e colunas com o mesmo nome das propriedades para realizar o mapeamento, mas esse comportamento pode ser modificado com as anotações \annotation{@Table}, \annotation{@Column}, \annotation{@Transient} e muitas outras. A propriedade \est{name} por exemplo, está decorada com a anotação \annotation{@Column} especificando que aquela propriedade não pode ser nula no banco de dados e a propriedade \est{birthDate} está decorada com a anotação \annotation{@Temporal} especificando que a coluna no banco deve ser do tipo \classe{Date}.

Na seção 3.1.3, durante a configuração do projeto, a propriedade \est{hibernate.hbm2ddl.auto} foi configurada com o valor \est{"create"}, assim o \est{Hibernate} cria as tabelas no banco de dados de acordo com as configurações de classes decoradas com a anotação \annotation{@Entity}. Outros valores para essa configuração podem ser \est{"update"}, para atualizar a estrutura das tabelas sem perder dados, e \est{"validate"}, para apenas verificar se o mapeamento objeto/relacional está consistente. O \est{Hibernate} cria tabelas, mas não cria o banco de dados, para que a criação de tabelas funcione é necessário que o banco de dados já exista. As tabelas são criadas no momento em que a aplicação é iniciada.

\subsection{\sigla{.NET}}

Entidades no projeto ASP.NET MVC são criadas na pasta \arquivo{Model/Entities}. Considere que a classe \classe{Person} é movida para essa pasta. O quadro~\ref{lst:34} mostra a classe \classe{BaseEntity} do projeto \est{ASP.NET MVC}.

\javacode{code/34.txt}{Classe \classe{BaseEntity} no projeto \est{ASP.NET}}{lst:34}

Uma convenção do \est{Entity Framework} é que se uma classe tiver uma propriedade chamada \est{Id} de tipo numérico ou \est{guid} (tipo de identificar único), essa propriedade será usada como chave primária. Assim nenhuma configuração adicional é necessária.

Assim como no projeto \lang{Java}, a classe \classe{Person} do projeto \est{ASP.NET MVC} é modificada para ser subclasse de \classe{BaseEntity}, como mostrado no quadro~\ref{lst:35}.

\javacode{code/35.txt}{Classe \classe{Person} no projeto \est{ASP.NET} agora herda de \classe{BaseEntity}}{lst:35}

Assim como no projeto \lang{Java}, entidades do \est{Entity Framework} podem ser decoradas com anotações que controlam como as tabelas serão geradas. No exemplo acima a anotação \annotation{Required} também diz que a coluna correspondente à propriedade \est{Name} não pode aceitar valores nulos.

\subsubsection{Criando o contexto do banco de dados}

Não existe configuração nas classe do projeto \est{ASP.NET MVC} que indique que elas são entidades e que devem ser usadas no mapeamento objeto/relacional. Esse papel é da classe \classe{DbContext} do \est{Entity Framework}. A classe \classe{DbContext} representa o banco de dados como um todo. Para criar o banco de dados a partir das entidades, o desenvolvedor deve criar uma classe que herde de \classe{DbContext} e tenha propriedades do tipo \classe{DbSet} como a classe \classe{NetWebCrudContext} no exemplo do quadro~\ref{lst:36}. 

\sharpcode{code/36.txt}{Classe \classe{NetWebCrudContext} representa o contexto do banco de dados}{lst:36}

A classe \classe{NetWebCrudContext} possui um \classe{DbSet} de objetos do tipo \classe{Person} que representa a tabela que é gerada no banco de dados. Para utilizar o banco, o desenvolvedor deve criar uma instância do contexto e fazer consultas aos \classe{DbSets} usando o Language Integrated Query (\lang{LINQ}). O \lang{LINQ} é uma linguagem para realizar consultas em coleções do \est{.NET Framework}. Usado em \classe{DbSets}, ele gera consultas \lang{SQL} automaticamente para manipular o banco de dados. Consultas com \lang{LINQ} são abordadas na seção 6.4.

O construtor de \classe{NetWebCrudContext} executa o construtor de \classe{DbContext} recebendo como parâmetro o nome da \est{connection string} que contém informações de conexão com o banco. Ele recebe o nome da \est{connection string} \est{NetWebCrudContext}, a mesma que foi configurada no arquivo \arquivo{Web.xml} na seção 3.2.2.

O construtor do contexto do banco de dados também é usado para configurar o tipo de inicialização do banco. O construtor de \classe{NetWebCrudContext} utiliza como inicializador a classe \classe{DropCreateDatabaseIfModelChanges}, então se qualquer entidade for modificada, o banco de dados é totalmente reconstruído e todos os dados perdidos. Existem outras classes que podem ser usadas como inicializadores como \classe{CreateDatabaseIfNotExists}, para criar o banco apenas uma vez, e \classe{DropCreateDatabaseAways} para sempre recriar o banco de dados quando a aplicação iniciar. O desenvolvedor pode estender esses inicializadores ou construir um completamente novo que implemente a interface \classe{IDatabaseInitializer}.

O \est{Entity Framework} não atualiza automaticamente a estrutura do banco de dados sem perder informações, para isso é necessário utilizar o \est{Entity Framework Migrations}. Essa funcionalidade permite utilizar as linguagens \lang{C\#} ou \lang{VB.NET} para executar \est{scripts} no banco de dados e atualizar sua estrutura. O \est{Entity Framework Migrations} não é abordado nesse trabalho, para mais informações recomenda-se consultar a documentação do \est{Entity Framework}.

O método \metodo{OnModelCreating} é usado para configurar diversos aspectos da criação do banco de dados.  No exemplo do quadro~\ref{lst:36}, a convenção de pluralizar o nome das tabelas é removida.

Diferente do \est{Hibernate} que cria ou atualiza o banco de dados quando a aplicação é iniciada, o \est{Entity Framework} cria o banco de dados na primeira vez em que uma instancia de \classe{DbContext} é criada.

Uma instancia de \classe{NetWebCrudContext} precisa ser instanciada para que se possa usar o banco de dados. No projeto exemplo, cada requisição de um usuário deve criar uma nova instancia de \classe{NetWebCrudContext}. Se mais de um usuário utilizar a mesma instancia, poderão haver erros de concorrência e um usuário poderá ter acesso os dados de outro que ainda não foram salvos para o banco de dados. O \est{Ninject} é responsável por criar as instancias de \classe{NetWebCrudContext} e injeta-las nos repositórios que serão criados na seção 6.3.2. O quadro~\ref{lst:37} mostra a configuração de injeção de \classe{NetWebCrudContext}.

\sharpcode{code/37.txt}{Injeção do contexto do banco de dados}{lst:37}

\section{Relacionamentos entre entidades}

Entidades podem se relacionar com outras com o uso de chaves estrangeiras. O relacionamento entre a classe \classe{Person} e a classe \classe{Contact} é usado nos exemplos dessa seção.

\subsection{\lang{Java}}

A classe \classe{Contact} é criada no pacote \pacote{br.uece.webCrud.model} como mostra o quadro~\ref{lst:38}.

\javacode{code/38.txt}{Classe \classe{Contact} no projeto \lang{Java}}{lst:38}

Assim como a classe \classe{Person}, a classe \classe{Contact} herda de \classe{BaseEntity} e é decorada com a anotação \annotation{@Entity}. O Relacionamento com a classe \classe{Person} é configurado pela anotação \annotation{@OneToOne} decorando o atributo \est{person}. O atributo \est{mappedBy} recebendo o valor \est{"contact"} quer dizer que na classe \classe{Person} deve existir um atributo com esse nome, configurado como o dono do relacionamento entre as tabelas. Resumindo, a tabela correspondente à classe \classe{Person} deve ter a coluna com a chave estrangeira para \classe{Contact}.

Além da anotação \annotation{@OneToOne}, o \sigla{JPA} possui as anotações \annotation{@OneToMany}, \annotation{@ManyToOne} e \annotation{@ManyToMany} para criar relacionamentos. As anotações \annotation{@OneToMany} e \annotation{@ManyToMany} devem ser usadas em uma coleção de objetos, quando uma entidade tem um relacionamento de "um para muitos" ou de "muitos para muitos" com outra entidade.

O relacionamento na classe \classe{Person} também deve ser configurado como mostra o quadro~\ref{lst:39}.

\javacode{code/39.txt}{Classe \classe{Person} com relacionamento para \classe{Contact}}{lst:39}

A classe \classe{Person} agora possui uma propriedade do tipo \classe{Contact}, também decorada com a anotação \annotation{@OneToOne} e com a anotação \annotation{@JoinColumn}. A primeira anotação é configurada para cascatear todas as operações de \classe{Person} para \classe{Contact}. Se por exemplo, um objeto do tipo \classe{Person} for atualizado no banco de dados, a sua propriedade \est{contact} também será atualizada. A configuração \est{fetch} recebe como valor \classe{FetchType.EAGER}, configurando a propriedade \est{contact} para ser recuperada do banco de dados por \est{eager loading} (executando uma única consulta). A propriedade \est{fetch} também pode receber \classe{FetchType.LAZY}, fazendo com que \est{contact} só seja carregado quando necessário, mas gerando uma nova consulta ao banco de dados.

A anotação \annotation{@JoinColumn} configura a classe \classe{Person} para criar a coluna da chave estrangeira na sua tabela correspondente, essa coluna tem o nome de \est{contactId}. O \est{Hibernate} detecta automaticamente as chaves decoradas com \annotation{@Id} e as utiliza como chaves estrangeiras em tabelas relacionadas. A Figura~\ref{fig:36} mostra o resultado de criação das tabelas no banco de dados \est{MySQL}. 

\figura{36.png}{Tabelas geradas pelo \est{Hibernate}}{fig:36}

\subsection{\sigla{.NET}}

A classe \classe{Contact}, ilustrada no quadro~\ref{lst:40}, também é criada no projeto \est{ASP.NET MVC}.

\sharpcode{code/40.txt}{Classe \classe{Contact} no projeto \est{ASP.NET MVC}}{lst:40}

A classe \classe{Contact} não requer nenhuma configuração, sendo apenas preciso herdar de \classe{BaseEntity} para adquirir a propriedade \est{Id}. Por outro lado, a classe \classe{Person} recebe mais propriedades como mostra o quadro~\ref{lst:41}.

\sharpcode{code/41.txt}{Classe \classe{Person} no projeto \est{ASP.NET} com novas propriedades}{lst:41}

Na classe \classe{Person} é adicionada uma propriedade \classe{Contact} modificada com a palavra chave \est{"virtual"} para configurá-la como uma propriedade de navegação que pode ser carregada por \est{lazy loading}. Isso resulta em mais uma consulta no banco, mas com o uso do \lang{LINQ} é possível recuperar as informações de \classe{Person} e \classe{Contact} com apenas uma consulta.

A propriedade \est{ContactId} gera a coluna da chave estrangeira. Por convenção do \est{Entity Framework}, propriedades de chaves estrangeiras devem ter o nome da propriedade de navegação (\classe{Contact}) seguido do nome da sua chave primária (\est{Id}).

Por último, é necessário adicionar um \classe{DbSet} para \classe{Contact} no contexto do banco de dados, como mostra o quadro~\ref{lst:42}.

\sharpcode{code/42.txt}{Classe \classe{NetWebCrudContext} com o \classe{DbSet} para \classe{Contact}}{lst:42}

O resultado no banco é semelhante ao do projeto \est{Java/Spring}, como mostra a Figura~\ref{fig:37}. Uma diferença que deve ser ressaltada é a criação da tabela \est{\_\_migrationhistory}. Essa tabela é usada pelo \est{Entity Framework} para armazenar dados sobre o histórico da estrutura do banco de dados. Assim o \est{Entity Framework Migrations} sabe quais modificações devem ser feitas no banco quando as entidades são modificadas.

\figura{37.png}{Tabelas geradas pelo \est{Entity Framework}}{fig:37}

\section{Repositórios}

A persistência e acesso aos dados são feitos utilizando o padrão de repositórios. Em cada projeto é criado um repositório genérico, que pode realizar operações básicas (salvar, atualizar, recuperar e apagar) em qualquer classe que seja subclasse de \classe{BaseEntity}. Um repositório especifico para a entidade \classe{Person} é criado nos dois projetos e outro para \classe{Contact} apenas no projeto \est{ASP.NET MVC}. Eles são especializações do repositório genérico e possuem operações especificas para essas entidades.

É demonstrado como utilizar a \est{Java Persistence Query Language} (\lang{JPQL}) e a \est{Language Integrated Query} (\sigla{LINQ}) para realizar consultas ao banco de dados e como gerenciar transações. Por último, os repositórios são injetados nas classes de serviço, substituindo as listas em memória.

\subsection{\lang{Java}}

Nos exemplos dessa seção, os repositórios e interfaces que eles implementam no projeto \est{Java/Spring} são criados no pacote \pacote{br.uece.webCrud.repositories}.

Primeiramente, são criadas a interface \classe{GenericRepository} e a classe \classe{GenericRepositoryImpl}, demonstradas nos quadros~\ref{lst:43} e~\ref{lst:44} respectivamente. Essa classe é o repositório genérico mencionado anteriormente.

\javacode{code/43.txt}{\est{Interface} \classe{GenericRepository}}{lst:43}

\javacode{code/44.txt}{Classe \classe{GenericRepositoryImpl}}{lst:44}

A classe \classe{GenericRepositoryImpl} é decorada com a anotação \annotation{@Repository}, então instâncias dela podem ser injetadas pelo \est{Spring IoC Container}. Erros de seus métodos em tempo de execução, dispararam a exceção \classe{DataAccessException} do \est{Spring Framework}.

Essa classe faz uso de \est{Java Generics}, podendo realizar suas operações com qualquer classe que herde de \classe{BaseEntity}. Seu construtor de extrai o tipo da classe com a qual ele está trabalhando a utiliza em suas operações. \classe{GenericRepositoryImpl} é uma classe abstrata (que não pode ser instanciada), quem efetivamente poderá executar suas operações serão os repositórios que herdarão dessa classe.

O objeto da classe que executa as operações no banco de dados é do tipo \classe{EntityManager}, decorado com a anotação \annotation{@PersistenceContext}. A instancia de \classe{EntityManager} é criada pelo \est{bean} \classe{LocalContainerEntitityManagerFactoryBean}, configurado na seção 3.1.3, e injetado pelo \est{Spring IoC Container}. Para objetos de acesso a dados, a anotação \annotation{@PersistenceContext} funciona do mesmo modo que \annotation{@Autowired}.

O método \metodo{getAll} retorna todas as entradas de uma determinada entidade armazenada no banco de dados. Ele mostra um exemplo de como criar uma consulta usando o método \metodo{createQuery} de \classe{EntityManager}, recebendo o texto da consulta escrito em \lang{JPQL}. Essa linguagem possui similaridades com \lang{SQL}. O método \metodo{getResultList} executa a consulta criada, convertendo o texto \lang{JPQL} em \lang{SQL}, realizando a consulta no banco de dados, mapeando tuplas para objetos, e devolvendo o resultado como uma lista. Para mais informações sobre o \lang{JPQL}, é recomendado consultar a documentação da linguagem no site \url{http://docs.oracle.com/cd/E15523_01/apirefs.1111/e13946/ejb3_langref.html}.

O método \metodo{findById} usa o método \metodo{find} do \classe{EntityManager} para encontrar uma entidade com chave primária e do tipo que foram passados como parâmetros. O método \metodo{add} chama o método \metodo{persists} para inserir um novo objeto no banco de dados. Esse objeto não pode ter uma chave primária de valor já existente no banco, ou o \est{Hibernate} irá disparar uma exceção \classe{PersistenceException}.

O método \metodo{merge}, chamado por \metodo{update}, realiza atualização dos valores de uma entidade no banco de dados. Para que ele funcione é necessário que a chave primária do objeto a ser atualizado exista, ou o \est{Hibernate} também disparará uma exceção. O método \metodo{delete}, usa \metodo{getById} para recuperar uma entidade pela sua chave primária e chama o método \metodo{remove} de \classe{EntityManager} para remove-la do banco de dados.

Com o repositório genérico criado, é possível criar um repositório especifico para a classe \classe{Person}, como mostram os quadros~\ref{lst:45} e~\ref{lst:46}. A classe \classe{PersonRepositoryImpl} adiciona o método \metodo{existsWithName}, que verifica a existência de uma pessoa com um certo nome. Nesse método é criada uma consulta mais complexa, que retorna o valor booleano \est{true}, se existir uma entidade no banco de dados com a propriedade \est{name} igual a passada como parâmetro e propriedade \est{id} diferente da passada como parâmetro. Esse exemplo também mostra como adicionar parâmetros às consultas, utilizando método \metodo{setParameter}.

\javacode{code/45.txt}{Interface PersonRepository}{lst:45}

\javacode{code/46.txt}{Classe PersonRepositoryImpl}{lst:46}

Por último, a lista em memória da classe \classe{PersonServiceImpl} é substituída por \classe{PersonRepository} e a anotação \annotation{@Transactional} é adicionada à classe, como mostra o quadro~\ref{lst:47}.

\javacode{code/47.txt}{\classe{PersonServiceImpl} utilizando \classe{PersonRepository}}{lst:47}

A anotação \annotation{@Transactional} do \est{Spring Framework} pode ser utilizada em classes ou métodos. Essa anotação configura o gerenciamento de transações com o banco de dados. Decorando um método, essa anotação faz com que uma única transação seja usada em todas as operações de acesso a dados dentro daquele método. Decorando uma classe, como no exemplo do quadro~\ref{lst:47}, a anotação aplica o seu efeito a todos os métodos da classe.  O \est{Spring} controla internamente a criação e reutilização de transações já existentes usando o \est{bean} \classe{JpaTransactionManager} também configurado na seção 3.1.3. Se um método decorado com essa anotação for chamado internamente por outro também decorado com ela, o método interno irá aproveitar a transação que já existe e não irá criar uma nova. Esse comportamento padrão pode ser modificado configurando atributos da anotação. 

Caso não ocorram erros nas operações com o banco de dados, as operações da transação são persistidas de forma atômica. Caso haja algum erro, as operações são revertidas e uma exceção é disparada. De toda forma, a transação é destruída automaticamente quando não é mais utilizada.

Como o relacionamento de \classe{Person} com \classe{Contact} configurado com \est{CascadeType.ALL} na seção 6.2.1, quando um objeto do tipo \classe{Person} é salvo ou atualizado no banco de dados, o seu contato também é. Então nos exemplos desse capitulo para \lang{Java}, não houve a necessidade de se criar um repositório para a classe \classe{Contact}.

\subsection{\sigla{.NET}}

Os repositórios no projeto \est{ASP.NET MVC} são criados na pasta \arquivo{Models/Repositories}. Assim como no projeto \est{Java/Spring}, começa-se criando o repositório genérico e sua \est{interface}. O quadro~\ref{lst:48} mostra a \est{interface} do repositório genérico, \classe{IGenericRepository}, e o quadro~\ref{lst:49} mostra sua implementação, \classe{GenericRepository}.

\sharpcode{code/48.txt}{\est{Interface} \classe{IGenericRepository}}{lst:48}

\sharpcode{code/49.txt}{Classe \classe{GenericRepository}}{lst:49}

Assim como no projeto \est{Java/Spring}, o repositório genérico é uma classe abstrata que realiza operações no banco de dados para classes que herdam de \classe{BaseEntity}. Também semelhante ao outro projeto, existe uma propriedade no repositório que efetivamente irá persistir dados e consultar o banco, a propriedade \est{context} do tipo \classe{NetWebCrudContext}. A propriedade \est{EntitySet} é usada para facilitar o acesso ao \classe{DbSet} correto da entidade a qual o repositório genérico irá manipular.

O método \metodo{GetById} mostra um exemplo de como se usar expressões \est{lambda} e a sintaxe fluente do \lang{LINQ} para realizar uma consulta. Métodos de filtragem e ordenação da sintaxe fluente do \lang{LINQ} retornam coleções do tipo \classe{IQueryable}, essa \est{interface} permite que mais métodos sejam chamados em cascata para a criação de consultas complexas. Uma consulta do \lang{LINQ} só é realmente executada no bando de dados quando se chamam certo métodos, como \metodo{Single}, \metodo{First}, \metodo{ToList} e \metodo{ToArray}, e essas consultas também são transformadas em \est{scripts} \lang{SQL} quando são executadas no banco de dados. No método \metodo{GetAll}, o método \metodo{ToList} do \classe{DbSet} retorna todas as tuplas do banco de dados transformadas em uma lista de objetos.

Dentro do método \metodo{Save}, um objeto é adicionado ao \classe{DbSet} usando o método \metodo{Add}, equivalente a uma operação de inserção na tabela. Operações de inserção, atualização e deleção de dados só são persistidas após a chamada do método \metodo{SaveChanges} do contexto. Esse método cria uma transação que em caso de erro reverte a transação e dispara uma exceção. É possível ter um controle melhor sobre transações, como ainda será visto nessa seção.

O método \metodo{Update} é um pouco mais complexo. Para persistir dados atualizados de uma entidade no banco de dados, caso a entidade não esteja anexada ao contexto, é necessário anexa-la e utilizar o contexto para modificar seu estado para \classe{Modfied}. Assim, quando \metodo{SaveChanges} for chamado, o contexto saberá que aquela entidade é de um objeto que já existe no banco de dados (através da chave primária já existente) e irá atualizar suas informações.

O método \metodo{Delete} é bastante similar ao do repositório do projeto \est{Java/Spring}. Uma entidade é recuperada pela sua chave primária e o método \metodo{Remove} do \classe{DbSet} a remove do banco de dados.

O método \metodo{GetTransaction} proporciona um melhor controle sobre transações. O controle automático de transações que o \est{Entity Framework} possui no método \metodo{SaveChanges} pode não ser suficiente para conversações extensas com o banco de dados em regras de negócio complexas. Então é possível criar uma transação manualmente, com o método \metodo{BeginTransation}, e utilizar essa transação para persistir ou reverter alterações. Mesmo com o uso dessa transação, e necessário chamar o método \metodo{SaveChanges} antes executar a persistência de dados.

Os repositórios para as entidades \classe{Person} e \classe{Contact} são criados como mostram os quadros~\ref{lst:50} e ~\ref{lst:51}. O \est{Entity Framework} cascateia automaticamente operações de inserção e deleção entre entidades relacionadas, mas não de atualização. Sendo assim, é necessária a existência de repositório de \classe{Contact} para que atualizações dessa entidade sejam persistidas.

\sharpcode{code/50.txt}{\est{Interfaces} \classe{IPersonRepository} e \classe{IContactRepository}}{lst:50}

\sharpcode{code/51.txt}{Classes \classe{PersonRepository} e \classe{ContactRepository}}{lst:51}

Assim como no projeto \est{Java/Spring}, o repositório para a classe \classe{Person} possui um método que verifica a existência de uma outra entidade que possua o mesmo nome. A consulta é feita com uma expressão \est{lambda} do \lang{LINQ}. O repositório para \classe{Contact} não possui nenhuma operação adicional, usa apenas as operações herdades de \classe{GenericRepository}. Como esses repositórios também serão injetados na classe de serviço \classe{PersonService}, sua configuração de injeção é adicionada à classe \classe{NinjectWebCommon}, como mostra o quadro~\ref{lst:52}.

\sharpcode{code/52.txt}{Configuração de injeção de repositórios no projeto \est{ASP.NET MVC}}{lst:52}

Por último, a classe de serviço \classe{PersonService} é atualizada como no quadro~\ref{lst:53}. O método \metodo{Update} mostra como fazer controle manual de transações. Chamando o método \metodo{GetTransaction} de \classe{PersonRepository}, ou de qualquer outro repositório, uma transação é criada para o contexto \classe{NetWebCrudContext} que é injetado no escopo da requisição do usuário. Dessa forma, só existe uma instância do contexto do banco de dados sendo utilizada por todos os repositórios, então a transação também é única.

Usando uma transação dessa forma, o desenvolvedor pode fazer as operações que desejar com quaisquer repositórios, e chamar o método \metodo{Commit} da transação para finalizar a persistência. Em caso de erros, é chamado o método \metodo{Rollback} para reverter às alterações no banco de dados. Como a transação está dentro do escopo da palavra chave \est{using}, o coletor de lixo do \est{.NET Framework} se encarrega de chamar o método \metodo{Dispose} automaticamente para finalizar a transação.

\sharpcode{code/53.txt}{Classe \classe{PersonService} utilizando ambos os repositórios no projeto \est{ASP.NET MVC}}{lst:53}

\section{Conclusão}

A criação de entidades para o modelo objeto/relacional com \est{Hibernate} e \est{Entity Framework} possuem abordagens distintas. No primeiro, classes precisam de anotações para serem tratadas como entidades, e no segundo não há essa necessidade. No \est{Entity Framework}, classes mais simples podem ser usadas e convenções cuidam de lógica no mapeamento objeto/relacional. A desvantagem do uso do \est{Entity Framework} é a necessidade de criação de uma classe de contexto que representa o banco de dados. Dependendo da complexidade da aplicação, a codificação e manutenção da classe de contexto pode se tornar uma inconveniência.

O relacionamento entre entidades no \est{Hibernate} é feito com anotações na própria entidade, e a configuração de relacionamentos de muito para muitos, por exemplo, pode se tornar complicada de entender para um desenvolvedor iniciante. O \est{Entity Framework} abstrai essa configuração, sendo apenas necessária a criação de propriedades de navegação entre entidades e propriedades que representem chaves estrangeiras.

Repositórios em projetos \est{Spring} com \est{Hibernate} utilizam uma implementação de \classe{EntityManager} para manipular entidades, podendo usar \est{Criteria Queries} ou \lang{JPQL} para gerar consultas. \est{Criteria Queries} utilizam vários métodos para gerar uma consulta, que no fim resulta em um código complexo, de baixa legibilidade e de manutenção questionável. A linguagem \lang{JPQL} se assemelha a \lang{SQL} e é mais legível, mas tem a desvantagem de ser escrita com \est{strings}, o que a torna mais propensa a erros. Em projetos com o \est{Entity Framework}, a linguagem \lang{LINQ} é usada para gerar consultas. Essa linguagem é de fácil entendimento, e pode utilizar as vantagens do \est{IntelliSense} (ferramenta de auto completar código do \est{Visual Studio}), tornando a detecção de erros mais fácil e a escrita de código mais eficiente.

Quanto ao controle de transações, o modo automático como o \est{Spring Framework} lida com transações o tornam uma melhor opção nesse aspecto do que o \est{Entity Framework}. 