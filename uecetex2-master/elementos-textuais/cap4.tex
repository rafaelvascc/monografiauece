\chapter{Criando \textit{controllers} e \textit{views}}

Neste capitulo é demonstrado como criar \textit{controllers} e \textit{views} em ambos os ambientes de desenvolvimento. São expostos exemplos de como criar \textit{controllers} com ações que retornam páginas, objetos \textit{JSON} e exemplos de ações que aceitam parâmetros. São dados também exemplos de como usar as \textit{view engines} que geram páginas \textit{HTML} dinamicamente.

\textit{view} é o termo inglês utilizado para se referir à camada de apresentação e/ou interface gráfica de uma aplicação. Em uma aplicação \textit{web}, \textit{views} são páginas mostradas no navegador do usuário.

Um \textit{controller} é uma classe que  recebe requisições, processa informações e devolve um resultado ao usuário.  Em uma aplicação \textit{web}, resultados comuns são páginas da \textit{web} e objetos \textit{JSON} ou \textit{XML} para processamento no navegador (usando Javascript). Eles também tem a função de selecionar e atualizar dados de \textit{views} e atualizar os dados da camada \textit{Model} \cite{06}.

\section{Criando um \textit{controller} com uma ação que retorna uma página}

Uma das funções mais comuns dos \textit{controllers} é retornar páginas para o usuário. Nas próximas seções é criado um \textit{controller} simples que irá retornar uma página estática para exibição no navegador.

Numa aplicação \textit{MVC}, um controller não deve instanciar nem usar métodos de outro \textit{controller}. Se o desenvolvedor achar que isso é necessário, é prudente considerar incluir a funcionalidade necessária em uma classe de serviços (ou classe de negócios). \textit{Controllers} devem apenas receber requisições, se for o caso chamar outras classes para processar requisições, e no fim devolver o resultado ao usuário. O capítulo 6 abordará a criação de classes de serviço.

Os exemplos desse capítulo irão manipular uma classe chamada \textit{Person}. Os quadros~\ref{lst:7} e~\ref{lst:8} mostram o código das classes em \textit{Java} e \textit{C\#}.

\javacode{code/7.txt}{Classe \textit{Person} em \textit{Java}}{lst:7}

A anotação \textit{@DateTimeFormat} (linha 7) especifica em qual formado apropriedade \textit{birthDate} será mostrada na \textit{view}. A escolha desse formato será explicada na seção 5.1.2.

\sharpcode{code/8.txt}{Classe \textit{Person} em \textit{C\#}}{lst:8}

\subsection{\textit{Java}}

\textit{Controllers} no \textit{Spring MVC} são classes decoradas com a anotação \textit{@Controller}. Essas classes podem estar contidas em qualquer pacote do projeto. No projeto exemplo, elas estarão contidas no \pacote{pacote br.uece.webCrud.controller}. O código contido no código fonte~\ref{lst:9} mostra um exemplo de um \textit{controller}, com o nome de \textit{PersonController}, e com uma ação que retorna uma página.

\javacode{code/9.txt}{\textit{PersonController} no projeto \textit{Spring}}{lst:9}

A anotação \textit{@RequestMapping} (linha 10) configura com qual o caminho, a partir da raiz da aplicação, as ações do \textit{controller} são acessadas. A mesma anotação em uma ação configura qual o caminho para acessá-la a partir do caminho do \textit{controller}. Por exemplo, para acessar a ação \textit{addPage} (linha 11) o usuário entraria com o endereço <raiz da aplicação>/person/add. É possível não usar o \textit{@RequestMapping} no \textit{controller} e usá-lo apenas na ação, mas fazendo isso o caminho para a ação será em relação à raiz da aplicação. Também é possível configurar para qual método \textit{HTTP} a ação irá responder. No exemplo acima, a ação é executada apenas para requisições usando o método \textit{GET}.

No \textit{Spring MVC} uma das maneiras de retornar uma página para o usuário é retornar um objeto \textit{String} com o caminho do arquivo da página. Lembrando que na seção 4.1.3 foram configurados um prefixo e um sufixo no objeto \textit{InternalResourceViewResolver}, e na seção 4.1 o \textit{Gradle} foi configurado com o nome da pasta onde ficará armazenado o conteúdo especifico da \textit{web}. Então o caminho completo para o arquivo que será retornado é <raiz do projeto>/WebContent/WEB-INF/person/add.jsp.

Agora é criada a página que essa primeira ação irá retornar. É criada uma pasta chamada \textit{person} dentro de \textit{WEB-INF} e dentro dessa pasta é criado o arquivo \textit{add.jsp}, como pode ser visto na Figura~\ref{fig:24}.

\figura{24.png}{Primeira \textit{view} adicionada ao projeto \textit{Java}}{fig:24}

O conteúdo do arquivo \textit{add.jsp} é editado como no exemplo do código fonte~\ref{lst:10}.

\jspcode{code/10.txt}{Arquivo \textit{add.jsp}}{lst:10}

A diretiva \textit{page} no cabeçalho da página (linha 1) diz ao servidor como tratar o arquivo, no caso como uma página \textit{HTML}. O resto do arquivo é um formulário \textit{HTML} comum, que recebe dois valores (o nome e a data de nascimento de uma pessoa) e tem um botão que os envia para o servidor. Note que o formulário faz o envio usando o método \textit{POST}.

\subsection{\textit{.NET}}

Um \textit{controller} é adicionado ao projeto \textit{ASP.NET MVC} clicando com o botão direito do mouse na pasta \textit{Controllers}, em seguida selecionado as opções \textit{"Add"} e \textit{"Controller"}. A Figura~\ref{fig:25} ilustra a adição do \textit{controller}.

\figura{25.png}{Adicionando um \textit{controller} no \textit{Visual Studio}}{fig:25}

São exibidas diversas opções de modelos de \textit{controllers}, a opção usada é a \textit{"MVC 5 Controller – Empty"} e o \textit{controller} foi também nomeado como \textit{PersonController}. Uma das convenções do \textit{ASP.NET MVC} é que \textit{controllers} sempre tem que terminar com a palavra \textit{"Controller"}. Normalmente é recomendado que se nomeie \textit{controllers} com o nome da entidade que eles manipulam como prefixo, para facilitar a organização do código.

Também é adicionada uma ação chamada \textit{Add} que retorna uma página. O código do \textit{controller} deve ficar como no código fonte~\ref{lst:11}. Note que \textit{PersonController} é subclasse da classe \textit{Controller}.

\sharpcode{code/11.txt}{\textit{PersonController} em \textit{C\#}}{lst:11}

A ação \textit{Add} retorna um objeto do tipo \textit{ActionResult}, que é uma classe pai de vários tipos de resultados no \textit{ASP.NET MVC} como \textit{JSONResult} e \textit{ViewResult}. O método \textit{View} retorna um \textit{ViewResult}, que é uma página \textit{web}. A ação \textit{Add} também está decorada com anotação \textit{HttpGet}, logo essa ação irá responder apenas ao método \textit{GET}.

No \textit{ASP.NET MVC}, o mapeamento de endereço para \textit{controllers} e ações é centralizado pela tabela de rotas, não por anotações isoladas em cada \textit{controller} e método. Quem configura a tabela de rotas é a classe \textit{RouteConfig}, mostrada no código fonte~\ref{lst:12}.

\sharpcode{code/12.txt}{A classe \textit{RouteConfig}}{lst:12}

A configuração da tabela de rotas começa ignorando qualquer requisição a arquivos com extensão \textit{.axd}, que são arquivos de recursos do \textit{ASP.NET} (linha 14). Logo em seguida, é configurada a rota padrão. O padrão {controller}/{action}/{id} determina que o acesso às ações deve seguir o caminho <raiz da aplicação>/<nome do controller>/<nome da ação>/<parâmetro opcional chamado "id"> (linha 18). Então, para acessar a ação \textit{Add} de \textit{PersonController}, o caminho é <raiz da aplicação>/person/add. Note que não é necessário digitar \textit{personController} no caminho, apenas o prefixo é necessário. A rota padrão vem configurada apontando para uma ação chamada \textit{Index} de um \textit{controller} chamado \textit{HomeController} se apenas o endereço da raiz da aplicação for requisitado (linhas 19 à 22). O desenvolvedor pode mudar esses valores. 

Para adicionar uma \textit{view}, se clica com o botão direito do \textit{mouse} em uma ação e então na opção \textit{"Add View..."} como mostrado na Figura~\ref{fig:26}. 

\figura{26.png}{Adicionando uma \textit{view} no \textit{ASP.NET MVC}}{fig:26}

Outra convenção do \textit{ASP.NET MVC} é que views devem ter o nome das ações que as retornam e ficar dentro de uma pasta com o nome do \textit{controller}, como na Figura~\ref{fig:27}. É usado o modelo \textit{"Empty"} na criação desta primeira \textit{view}. A extensão de arquivos de views em projetos \textit{ASP.NET MVC 5} é \textit{.cshtml}.

\figura{27.png}{\textit{View} adicionada ao projeto \textit{ASP.NET MVC}}{fig:27}

A pasta \textit{Shared} pode conter \textit{views} acessíveis à todos os \textit{controllers}. Em alguns modelos de projetos do \textit{Visual Studio} ela contém um arquivo chamado \textit{\_Layout.cshtml}. Esse arquivo é uma página modelo (ou \textit{layout}) usada quando se quer aproveitar uma mesma estrutura para várias \textit{views} diferentes. Essa pasta também pode armazenar \textit{Partial Views}, que são pedaços de \textit{views} e componentes que o desenvolvedor pode usar em diversas páginas. \textit{Layouts} e \textit{Partial Views} não são abordados nesse trabalho.

O conteúdo do arquivo \textit{Add.cshtml} é editado para que fique igual ao exemplo do código fonte~\ref{lst:13}.

\razorcode{code/13.txt}{O arquivo \textit{Add.cshtml}}{lst:13}

Tanto no projeto \textit{Java/Spring} como no projeto \textit{ASP.NET MVC}, quando acessada pelo Google Chrome, a view é exibida como na Figura~\ref{fig:28}. 

\figura{28.png}{Resultado de exibição da \textit{view}}{fig:28}

\section{Criando ações parametrizadas}

Na seção 5.1, foram criadas ações que respondem ao método \textit{HTTP GET} e retornam uma página ao usuário. Nesta seção, essa página será utilizada para enviar informações ao servidor utilizando o método \textit{HTTP POST}. Nas \textit{views} de exemplo, o atributo \textit{action} não está definido nos formulários, então ele vai enviar informações ao servidor para o mesmo caminho usado para acessar a página que o contém (<raiz da aplicação>/person/add).

Devem ser criadas ações nos \textit{controllers} que respondam ao mesmo endereço da página \textit{Add}, mas usem método \textit{POST}. Essas ações também devem receber as informações que o formulário irá enviar. Isso é feito configurando anotações e adicionando parâmetros às ações em ambos os projetos. Nos exemplos das seções 5.2.1 e 5.2.2 são feitas ações que recebem um objeto do tipo \textit{Person} e o adicionam à uma coleção armazenada em memória.

\subsection{\textit{Java}}

O código presente no código fonte~\ref{lst:14} é adicionado à classe \textit{PersonController}:

\javacode{code/14.txt}{Ação de \textit{PersonController} no projeto \textit{Java} responsiva ao método \textit{POST}}{lst:14}

A anotação \textit{@RequestMapping} (linha 3) no método \textit{addPost} aponta para o mesmo caminho que o método \textit{addPage} mas ela responde às chamadas usando o método \textit{POST}. O método \textit{addPost} recebe como parâmetro um objeto do tipo \textit{Person}. Os parâmetros de métodos também podem ser decorados com anotações.

A anotação \textit{@ModelAttribute} (linha 4) permite que o \textit{Spring MVC} procure e mapeie valores do cabeçalho \textit{HTTP} para o objeto \textit{person}. No código fonte~\ref{lst:10}, o atributo \textit{HTML} \textit{name} dos elementos \textit{input} onde são digitadas as informações de um novo objeto são \textit{"name"} e \textit{"birthDate"} (linhas 12 e 13), os mesmos nomes dos atributos da classe \textit{Person}.

A Figura~\ref{fig:29} mostra as ferramentas de desenvolvedor do \textit{Google Chrome} com um exemplo do que acontece quando se envia os dados da página ao servidor.

\figura{29.png}{Enviando dados para o servidor}{fig:29}

No código fonte~\ref{lst:7}, o atributo \textit{birthDate} da classe \textit{Person} foi decorado com a anotação \textit{@DateTimeFormat} para que aceite o formato de data “yyyy-MM-dd”. A Figura~\ref{fig:29} mostra que esse é o formato de data que o \textit{Google Chrome} envia para o servidor. Se o formato de data não estivesse configurado ou for enviado em outro padrão, o servidor enviaria uma reposta de erro 400 (\textit{Bad Request}).

Quando chegam ao servidor, os dados são mapeados automaticamente para o parâmetro \textit{person} do método \textit{addPost} como mostrado na Figura~\ref{fig:30}. E a partir daí, o desenvolvedor pode trabalhar com os dados.

\figura{30.png}{Objeto \textit{person} com parâmetros corretamente populados no projeto \textit{Java}}{fig:30}

No código fonte~\ref{lst:14} retorno do método \textit{addPost} redireciona a requisição para outra ação que mostra uma página com uma tabela com todas pessoas adicionadas à lista (linha 11). As implementações dessa ação e página serão demonstradas em seções posteriores.

\subsection{\textit{.NET}}

O código apresentado no código fonte~\ref{lst:15} foi adicionado ao \textit{PersonController}.

\sharpcode{code/15.txt}{\textit{PersonController} do projeto \textit{ASP.NET} com nova ação}{lst:15}

O \textit{ASP.NET MVC} não guarda o estado de objetos entre requisições, então a lista deve ser guardada como uma variável da aplicação (linhas 1 à 12). Armazenada dessa forma, o estado da lista se preserva e está disponível para todos os usuários. Lembrando que em capítulos posteriores, a lista será substituída por serviços que persistem as entidades no bando de dados.

Por convenção do \textit{ASP.NET MVC}, a ação do \textit{controller} que reponde ao método \textit{POST} deve ter o mesmo nome da \textit{view} a qual ele responde.  A única configuração necessária é decorar a ação com a anotação \textit{HttpPost}. O \textit{ASP.NET MVC} faz o mapeamento dos dados para o objeto \textit{person} de forma semelhante ao \textit{Spring MVC}, só que não é necessário especificar o formato do campo \textit{birthDate}. A Figura~\ref{fig:31} mostra o resultado do mapeamento.

\figura{31.png}{Objeto \textit{person} no \textit{ASP.NET MVC} corretamente populado}{fig:31}

Assim como no projeto \textit{Java/Spring}, o objeto \textit{person} é adicionado em uma lista e a ação redireciona o usuário para uma página que mostra todos os objetos da lista.

\section{Retornando um objeto JSON}

Aplicações web modernas utilizam a técnica \textit{AJAX} para atualizar partes de páginas em vez de enviar uma requisição para que o servidor envie uma página completa novamente. Essa técnica é essencial para dar ao usuário uma melhor experiência de uso da aplicação. Os dados são enviados e recebidos assincronamente utilizando código \textit{Javascript}. Esses dados podem estar no formato \textit{XML} ou \textit{JSON}. Nesta seção, é demonstrado como retornar dados no formato \textit{JSON}. Como exemplo é retornado um único objeto do tipo \textit{Person} para o usuário.

Para mais informações sobre como manipular dados com \textit{Javascript} e \textit{AJAX}, é recomendada a leitura dos livros \textit{Pro Javascript For Web Apps} \cite{20} e \textit{Pro jQuery} \cite{21} da editora \textit{Apress}.

\subsection{\textit{Java}}

A biblioteca \textit{Jackson Databind} é responsável por converter objetos \textit{Java} para o formato \textit{JSON}. Essa biblioteca foi adicionada ao projeto durante a configuração do arquivo \textit{build.gradle} na seção 4.1.2. 

Para uma ação retornar um objeto \textit{JSON}, ela deve retornar um objeto \textit{Java} decorado com a anotação \textit{@ResponseBody}, como no código fonte~\ref{lst:16}.

\javacode{code/16.txt}{Ação no \textit{Spring MVC} que retorna um objeto \textit{JSON}}{lst:16}

Acessando a ação \textit{getOne} pelo navegador, é obtida a resposta exposta na Figura~\ref{fig:32}.

\figura{32.png}{Resultado \textit{JSON} no projeto \textit{Java}}{fig:32}

O atributo \textit{birthDate} é enviado ao navegador como um número que pode ser usado para criar um objeto \textit{Date} do \textit{Javascript}.

\subsection{\textit{.NET}}

O \textit{ASP.NET MVC} pode trabalhar com objetos no formato \textit{JSON} sem precisar de uma biblioteca de terceiros. O exemplo do código fonte~\ref{lst:17} mostra uma ação que retorna um objeto \textit{JSON}.

\sharpcode{code/17.txt}{Ação no \textit{ASP.NET MVC} que retorna um objeto \textit{JSON}}{lst:17}

O método \textit{Json} no final da ação faz a conversão de objetos para o formato \textit{JSON}. Por padrão o \textit{ASP.NET MVC} só envia objetos \textit{JSON} usando o método \textit{HTTP POST}, para a habilitar o uso de \textit{HTTP GET} é necessário um parâmetro adicional (linha 10). A Figura~\ref{fig:33} mostra a resposta à ação \textit{GetOne}.

\figura{33.png}{Resultado \textit{JSON} no projeto \textit{ASP.NET MVC}}{fig:33}

Aqui existe uma inconveniência. O modo como o \textit{ASP.NET} envia datas faz com que seja necessário mais código \textit{Javascript} para extrair o número que realmente representa a data e criar um objeto \textit{Date}.

\section{Gerando \textit{Views} Dinâmicas}

Nesta seção é demonstrado como fazer uma página que exibe todos os itens das listas de objetos \textit{Person} em ambos os projetos. São feitas ações que enviam as listas para as \textit{views} que montam o código \textit{HTML} usando as \textit{view engines} padrão de cada tecnologia, \textit{JSP Standart Tag Library} (\textit{JSTL}) no projeto \textit{Java/Spring} e \textit{Razor} no projeto \textit{ASP.NET MVC}.

\subsection{\textit{Java} (\textit{JSTL})}

Observe o método \textit{listPage} na classe \textit{PersonController} no exemplo do código fonte~\ref{lst:18}.

\javacode{code/18.txt}{Ação no projeto \textit{Java} que retorna uma página com a lista de \textit{Persons}}{lst:18}

O tipo de retorno do método \textit{listPage} e um objeto do tipo \textit{ModelAndView} (linha 8). Esse objeto representa uma página dinâmica que pode conter e manipular objetos \textit{Java}. O método é mapeado para o endereço <raiz da aplicação>/person/list (linha 1). 

A ação começa com a verificação de existência e, se for o caso, criação da lista. Logo em seguida ele cria um objeto \textit{ModelAndView} que aponta para uma nova página que será criada dentro da pasta \textit{person}, \textit{list.jsp}. O método \textit{addObject} adiciona a lista \textit{persons} para que a página possa utilizá-la com o nome de referência \textit{personsList}. O Resultado então é retornado para o usuário. O código fonte~\ref{lst:19} mostra o conteúdo do arquivo da página \textit{list.jsp}.

\jspcode{code/19.txt}{Código da página \textit{list.jsp}}{lst:19}

A página começa com o cabeçalho padrão de páginas \textit{jsp}. Logo abaixo, o cabeçalho \textit{taglib} adiciona a referência à biblioteca \textit{JSTL} básica (linha 3). Essa biblioteca contém um conjunto de \textit{tags} \textit{JSTL}, pedaços de código \textit{Java} que ajudam a montar páginas dinâmicas. As \textit{tags} \textit{JSTL} são escritas na página como pedaços de código que lembram \textit{tags} \textit{HTML}, mas devem começar utilizando prefixo configurado no cabeçalho \textit{taglib} (no caso do exemplo acima, a letra "c"). Até o fim da \textit{tag} \textit{HTML} <thead>, o que está escrito na página é apenas \textit{HTML} puro (linhas 5 à 20). A parte onde o \textit{JSTL} entra em ação é dentro do corpo da tabela (linha 21). 

A \textit{tag} \textit{forEach} cria um pedaço de bloco \textit{HTML} para cada item em uma coleção. Essa coleção é configurada no atributo \textit{items} e usando a sintaxe do \textit{JSTL} é dito que a coleção se chama \textit{personList}, o mesmo nome que foi usado como referencias para passar a lista de objetos \textit{Person} para a página (código fonte~\ref{lst:18}, linha 9). O atributo \textit{var} configura um nome de referência para o objeto que é usado em cada iteração da lista.

Dentro de cada iteração é criada uma nova linha para a tabela e, em cada linha, são criadas duas colunas: uma mostrando o atributo \textit{name} e outra o \textit{birthDate} do objeto \textit{person}. Note que não é necessário chamar os métodos \textit{getName} e \textit{getBirthDate}. É necessário apenas referenciar o nome do atributo que o desenvolvedor deseja mostrar na página. Depois de adicionar alguns objetos à lista, é obtido o resultado exposto na Figura~\ref{fig:34}.

\figura{34.png}{Resultado da página \textit{list.jsp}}{fig:34}

O \textit{JSTL} possui muitas outras \textit{taglibs}. Para mais informações, consulte a documentação do \textit{JSTL} \cite(22).

\subsection{\textit{.NET} (\textit{Razor})}

O exemplo do código fonte~\ref{lst:20} mostra a ação que retorna a página que exibe o conteúdo da lista.

\sharpcode{code/20.txt}{Ação que retorna a página \textit{list.cshtml}}{lst:20}

O método \textit{View} pode receber como parâmetro qualquer objeto que o desenvolvedor queira passar para a página, aqui ele recebe a lista de objetos do tipo \textit{Person} (linha 4). O código fonte~\ref{lst:21} mostra a página \textit{list.cshtml}.

\sharpcode{code/21.txt}{Código fonte da página \textit{list.cshtml}}{lst:21}

Além da sintaxe da \textit{view engine Razor} (que mistura código \textit{C\#} com \textit{HTML}) uma diferença em relação à página do projeto \textit{Java} é que uma página no \textit{ASP.NET MVC} pode ser fortemente tipada.

A primeira linha da página importa o \textit{namespace} \pacote{NetWebCrud.Models} para se utilizar de suas classes. A segunda linha informa qual é o tipo de objeto que será o modelo da página, uma lista de objetos \textit{Person} como a que foi passada na ação do \textit{controller} (código fonte~\ref{lst:20} linha 4).

O objeto \textit{Model} dentro do laço \textit{foreach} é a referência ao objeto que foi enviado para a página. \textit{Model} faz referência a qualquer tipo de objeto e a diretiva \textit{@model} no começo da página especifica seu tipo para que os recursos de auto completar do \textit{Visual Studio} estejam disponíveis e o \textit{ASP.NET MVC} possa gerar a página. O resultado pode ser visto na Figura~\ref{fig:35}.

\figura{35.png}{Resultado da página \textit{list.cshtml}}{fig:35}

Note que o \textit{ASP.NET MVC} exibe também a hora do atributo \textit{BirthDate}.

\section{Conclusão}

Enquanto no \textit{Spring MVC} o desenvolvedor deve decorar seus \textit{controllers} e ações com várias anotações \textit{@RequestMapping}, no \textit{ASP.NET MVC} o mapeamento de endereços é centralizado inteiramente na tabela de rotas. Isso junto com as convenções do \textit{ASP.NET MVC} ajudam o desenvolvedor a escrever menos código.

Os \textit{controllers} no \textit{Spring MVC} podem ser qualquer classe decorada com a anotação \textit{@Controller}, no \textit{ASP.NET MVC} devem ser classes que herdem da superclasse \textit{Controller} e tenham o sufixo \textit{"Controller"} no seu nome. O modo de criação de \textit{controllers} no \textit{Spring MVC} é mais flexível, mas é uma boa prática também adicionar o sufixo \textit{"Controller"} em tais classes e deixá-las em um pacote específico para \textit{controllers} objetivando uma melhor organização de código.

Ao enviar dados para o servidor, a única diferença significativa é o modo de enviar datas. O mapeamento dos dados da requisição \textit{HTTP} para objetos é simples e funcional em ambas as tecnologias.

O \textit{ASP.NET MVC} dá suporte nativo para manipulação de objetos no formato \textit{JSON}, bastando apenas que o desenvolvedor utilize o método \textit{Json} para gerar um objeto \textit{JSONResult}. O \textit{Spring MVC} não vem com suporte nativo para trabalhar com objetos \textit{JSON}, sendo necessária a biblioteca \textit{Jackson Databind} e a configuração de anotações.

A \textit{view engine} \textit{JSTL} utiliza uma sintaxe parecida com o \textit{HTML}, se integrando melhor ao código da página, enquanto o \textit{Razor} tem a vantagem da possibilidade de ser fortemente tipado e utilizar as vantagens do \textit{C\#} como a função de auto completar código do \textit{Visual Studio}. Em ambas as tecnologias o desenvolvedor pode estender as \textit{view engines}. Em projetos \textit{Java} novas \textit{taglibs} podem ser criadas e no \textit{ASP.NET MVC} podem ser feitas \textit{View Helpers} e \textit{Partial Views}. Em se trantando de \textit{views}, uma vantagem do \textit{ASP.NET MVC} em relação ao \textit{Java Enterprise Endition} é o suporte nativo à \textit{layouts} (também conhecidas como \textit{master pages}). Caso o desenvolvedor \textit{Java} queria usar \textit{layouts}, deve recorrer às bibliotecas de terceiros como o \textit{Sitemesh} ou utilizar outras \textit{view engines} como o \textit{FreeMarker}. E aqui se encontra a maior vantagem do \textit{Java Enterprise Edition} em relação à \textit{views}, o desenvolvedor pode escolher simplesmente utilizar outras \textit{view engines}.

No próximo capítulo são abordados a criação de classes de negócio/serviços e injeção de dependências.