\documentclass[a4paper,12pt]{article}
\usepackage[brazilian]{babel}
\usepackage[section]{placeins}
\usepackage[utf8]{inputenc}
\usepackage[T1]{fontenc}
\usepackage{indentfirst} %Margem no primeiro paragrafo
\usepackage{hyperref} %Urls
\usepackage[left=3cm,top=3cm,right=2cm,bottom=2cm]{geometry}
\usepackage{graphicx} %figuras
\usepackage{helvet} %fonte arial
\usepackage{setspace}
\usepackage{xspace}
\usepackage{xcolor}
\usepackage{listings}
\usepackage{minted}
\setlength{\parskip}{1em}
\renewcommand{\familydefault}{\sfdefault}
\renewcommand{\lstlistingname}{Código Fonte}% Listing -> Algorithm
\renewcommand{\lstlistlistingname}{Lista de Códigos Fonte}
\graphicspath{ {figuras/} }

\newcommand{\anmvc} {
	\sigla{ASP.NET MVC} 5
}

\newcommand{\spring} {
	\lang{Java}\textbackslash \est{Spring} \sigla{MVC}
}

\newcommand{\mysql} {
	\est{MySQL}
}

\newcommand{\figura}[2] {
	\begin{figure}[ht]
		\centering
		\includegraphics{#1}
		\caption{#2}
	\end{figure}
	\FloatBarrier
}

\newcommand{\est}[1] {
	\textit{#1}
}

\newcommand{\classe}[1] {
	\textit{#1}
}

\newcommand{\arquivo}[1] {
	\textit{#1}
}

\newcommand{\sigla}[1] {
	\textit{#1}
}

\newcommand{\lang}[1] {
	\textit{#1}
}

\newcommand{\groovycode}[1] {
	\inputminted{groovy}{#1}
	%\lstinputlisting[language=Java, caption=#2, breaklines=true, basicstyle=\ttfamily\fontsize{9}{9}]{#1}
}

\newcommand{\javacode}[1] {
	\inputminted{java}{#1}
}

\newcommand{\sharpcode}[1] {
	\inputminted[csharp]{#1}
}

\begin{document}
\title{Comparação de desenvolvimento de aplicações web com ASP.NET MVC 5 e Spring MVC 4}
\author{José Rafael Vasconcelos Cavalcante}
\date{January 5, 2011}
\maketitle

\section{Introdução}

\section{Configuração de ambiente de desenvolvimento}

Neste capitulo será demonstrado o preparo do ambiente de desenvolvimento em um computador rodando o sistema operacional \est{Windows} 8.1 de 64 \est{bits}. Primeiramente, instala-se um sistema gerenciador de banco de dados para trabalhar tanto com a plataforma \spring quanto com a plataforma \anmvc.  O banco de dados usado será o \est{MySQL Community Server} versão 5.6.22 (a versão mais atual até o momento de criação desta monografia). A \est{Integrated Development Environment} (\sigla{IDE}) utilizada  para escrever código em \lang{Java}, será o \est{Eclipse Luna}. O \est{Gradle} será usado como \est{build tool} e o \est{Apache Tomcat} como \est{container} \est{Java Enterprise Edition} (\sigla{JEE}). Para desenvolver em \lang{C\#}, será usado o \est{Visual Studio 2013 Community}. Considerando que o leitor já possui entendimentos sobre informática necessários para instalar programas no \est{Windows}, as instruções de instalação serão sucintas. 

\subsection{Instalação do MySQL}

O download do instalador do MySQL Community foi feito no seguinte endereço \url{https://dev.mysql.com/downloads/windows/installer/}5.6.html. O instalador está disponível duas versões, web installer e off-line installer. O web installer é um arquivo pequeno que quando executado irá baixar os arquivos do MySQL para a máquina, o off-line installer  é maior e vem com todos os arquivos necessários para a  instalação do MySQL. Qualquer que seja o método de instalação escolhido, eles terão as mesmas opções.

Executando o instalador, é escolhida a opção Custom e na árvore de opções que aparecerá na tela a seguir, são escolhidos o MySQL Server, o MySQL Workbench, o Connector/J (para Java) e o Connector/NET (para .NET), como mostrado na figura 1. Pode acontecer do instalador pedir para instalar o Microsoft Visual C++ 2013 como dependência do MySQL Workbench, se isso acontecer, o próprio instalador proverá um botão para instalar essa dependência.

\figura{mysqlinstaller1.jpg}{Opções de instalação do \mysql}

Concluída a instalação, é hora de configurar o serviço do MySQL. Deixa-se selecionado o tipo de configuração como Development Machine e as configurações de rede padrão (protocolo TCP/IP, porta 3306). Quando for necessária a senha do usuário Root, será usada “1234”, é uma senha fraca que não se recomenda usar em ambiente de produção, mas serve para propósito de exemplo. Finaliza-se a configuração deixando marcados os restantes das opções de configuração como padrão do instalador.

Com o objetivo de testar o sucesso da instalação, o desenvolvedor pode executar o MySQL Workbench, como ilustrado na figura 2,  e tentar se conectar à instancia do MySQL.

\figura{mysqlinstaller2.jpg}{O \mysql \est{Workbench}}

Para mais informações sobre o MySQL, visite a página oficial do projeto, \url{https://www.mysql.com/}.

\subsection{Preparando o ambiente Java}

Para desenvolver em Java, será utilizado o  Eclipse Luna e o Java Development Kit 8 (JDK 8). Será usado o Gradle como build tool através de um plugin do Eclipse e o servidor web utilizado será o Apache Tomcat.

\subsubsection{Instalando JDK 8}

O instalador do JDK 8 pode ser adquirido no endereço \url{http://www.oracle.com/technetwork/java/javase/downloads/jdk8-downloads-2133151.html}. Existem diversas versões para diversos sistemas operacionais, será usado nesse trabalho a versão para Windows de 64 bits. 

Para fazer a instalação do JDK, foi executado o arquivo de instalação seguindo  as suas instruções. A única configuração possível durante a instalação é a mudança da sua pasta de destino, mas será mantido o diretório padrão como mostrado na figura 3.

\figura{jdk81.jpg}{O instalador do JDK 8}

Terminada a instalação, é necessário configurar a variável PATH para que o sistema encontre os arquivos do Java. Essas opções de configuração estão no painel de controle do Windows, no caminho Sistema/Configurações avançadas do sistema/Variáveis de ambiente. Na janela de variáveis do sistema, é editada a variável PATH. Se adiciona o caminho onde o JDK foi instalado acrescido da pasta bin (C:\textbackslash Program Files\textbackslash Java\textbackslash jdk\textmd{1.8.0\_25}\textbackslash bin) como mostrado na figura 4.

\figura{path1.png}{Configurando a variável PATH}

Para testar se tudo foi instalado corretamente, abre-se uma janela do prompt de comando e digita-se o comando “java –version” (sem aspas). Se não existir problemas, será exibida na tela o número da versão do JDK instalado. Caso isso não aconteça, é aconselhável desinstalar o JDK e repetir o processo de instalação. 

\subsubsection{Instalando o Eclipse Luna} 

O Eclipse Luna para Desenvolvedores Java EE é encontrado no endereço \url{https://www.eclipse.org/downloads/}. Terminando o download, a pasta “eclipse” pode ser descompactada para qualquer diretório do computador. Coloca-se um atalho na sua área de trabalho para o executável do Eclipse (eclipse.exe) para facilitar o acesso.

Na primeira vez que o Eclipse é executado será exibida uma janela para configurar o Workspace padrão (uma pasta onde serão guardados projetos e configurações).

\figura{eclipse1.png}{Eclipse recém instalado}

\subsubsection{Instalando o plugin do Gradle} 

O Gradle é a build tool que será utilizada nos exemplos do projeto Java. Ele faz o mesmo trabalho que o ANT junto com o Ivy ou o Maven fazem, mas ele é considerado por alguns autores como o que existe de mais moderno em se tratando de build tools, pois ele usa a linguagem Groovy em vez de XML para os scripts de builds, e permite configuração que o Maven não permite. O Gradle está presente em todo ciclo de vida do software (ele gera artefatos, roda teste unitários, resolve dependências e executa integração continua), mas será usada apenas uma pequena parte do que ele pode oferecer. Para mais informações sobre o Gradle acesse \url{https://www.gradle.org/}.

No Eclipse Marketplace (repositório de plugins do Eclipse), faz-se uma perquisa por “Gradle” na barra de buscas. Entre os resultados está o Gradle IDE Pack, esse plugin será usado nos exemplos desse trabalho. O Eclipse pede confirmação de todos os pacotes necessários para a instalação, todos são selecionados e instalados. Aceita-se os termos de uso do Gradle e quando aparecer uma janela de alerta confirmando a instalação, clica-se em OK. Quando a instalação terminar, o Eclipse é reiniciado.

Para verificar se o plugin foi instalado com sucesso, a pasta Gradle deve aparecer na arvore de tipos de projetos, no menu de novos projetos.

\figura{plugin1.png}{Instalando o plugin do Gradle}

\subsubsection{Instalando o Apache Tomcat 8 como servidor de desenvolvimento do Eclipse} 

O Java EE é um conjunto de especificações que precisam ser implementadas por um container (um servidor de aplicação ou servidor web) que irá efetivamente executar a aplicação. Existem diversos containers disponíveis no mercado, sendo o GlassFish o próprio container da Oracle. Por motivos de praticidade, será usado o Apache Tomcat. O Tomcat pode ser gerenciado pela aba de servidores do Eclipse.

O instalador do Tomcat 8 pode ser encontrado no endereço \url{https://tomcat.apache.org/download-80.cgi}. Para os exemplos, usa-se a distribuição para Windows de 64 bits no formato zip. A pasta apache-tomcat-8.0.15 pode ser descompactada para qualquer diretório no disco rígido (como exemplo será usada a raiz do disco C:). 

A na aba “Servers” do Eclipse possui um link auto descritivo para adicionar um novo servidor. Clicando no link e expandindo pasta “Apache”, é escolhido o Tomcat 8 na árvore de opções. Na janela seguinte, em “Tomcat installation directory”, o botão “browse\ldots” é usado para escolher o caminho de instalação do Tomcat (C:\textbackslash apache-tomcat-\textmd{8.0.15} no nosso exemplo). Clicando no botão “Finish”, o Tomcat está pronto para uso com o Eclipse.

\figura{tomcat1.png}{Janela pra adicionar servidores no Eclipse}

\subsection{Instalando o Visual Studio Community 2013} 

O instalador do Visual Studio Community 2013 pode ser encontrado no endereço \url{http://www.visualstudio.com/en-us/news/vs2013-community-vs.aspx}. Esse é um instalador online, ele irá baixar os arquivos do Visual Studio e opcionais selecionados à medida que a instalação for progredindo. Uma imagem do DVD de instalação off-line também está disponível na sessão de downloads do site \url{http://www.visualstudio.com}.

\figura{vs1.png}{Instalador do Visual Studio Community 2013}

Clicando no botão “Next”, aa próxima tela o instalador irá exibir uma lista de componentes opcionais como o SDK do Windows Phone 8 e o Silverlight. Desses componentes opcionais, é aconselhável instalar pelo menos o Microsoft Web Developer Tools para facilitar o desenvolvimento de aplicações web.

Após a instalação, o desenvolvedor pode utilizar sua conta da Microsoft como perfil no Visual Studio e publicar suas aplicações no Microsoft Azure, mas isso é opcional. Quando se executa o Visual Studio pela primeira vez, o desenvolvedor pode escolher as opções de desenvolvimento e um esquema de cores que irá usar. Como exemplo, é escolhida a opção de desenvolvimento "Web Development”.

\figura{vs2.png}{Tela inicial do Visual Studio Community 2013}

O Visual Studio possui sua própria ferramenta de geração de builds (MsBuild), gerenciador de pacotes para obter bibliotecas de terceiros (Nuget) e um servidor de desenvolvimento minimalista para executar e depurar aplicações web. 

\subsection{Conclusão}

A preparação de um ambiente de desenvolvimento Java requer mais passos, instalar o kit de desenvolvimento do Java, uma IDE (Eclipse), um servidor Java EE (Tomcat) e uma build tool (Gradle), enquanto o todo o software necessário para se desenvolver com .NET é adquirido em um único instalador.

As vantagens do ambiente Java é que ele dá ao desenvolvedor mais opções de como configurar seu ambiente, e o tamanho em megabytes do software necessário é consideravelmente menor do que o ambiente .NET. O desenvolvedor tem a liberdade de escolher outras build tools disponíveis no mercado (Ex: Ant, Maven), outras IDEs (Ex: Netbeans) e outros servidores Java EE (Ex: Jetty, Glassfish). O lado “negativo” dessa liberdade é que com tantas opções, o desenvolvedor tem que pesquisar mais sobre cada solução até decidir como vai montar seu ambiente de desenvolvimento, e depois de escolher quais produtos irá utilizar, pode ser que precise mais algum tempo estudando como eles interagem.

A vantagem do ambiente .NET é a facilidade de se ter todo o software necessário para o desenvolvimento em um único pacote, o .NET Framework, Visual Studio Community 2013 e demais ferramentas. A desvantagem é que esse pacote pode conter coisas que o desenvolvedor não precisa ou deseja, baixando arquivos desnecessários e tomando espaço em disco.

No próximo capitulo será mostrado como criar um projeto de uma aplicação web nas duas plataformas.

\section{Criando projetos web no padrão MVC}

Nesse capitulo será mostrado como se cria um novo projeto para uma aplicação web que utiliza o padrão MVC nas plataformas Java e .NET. Para Java será usado o Spring Framework, que cuidará da arquitetura MVC e injeção de dependências, e o Hibernate, que cuidará da persistência e acesso a dados. Na plataforma .NET será utilizado o ASP.NET MVC 5, que cuida de toda estrutura de uma aplicação web MVC, o Entity Framework 6 para acesso a dados e o Ninject para injeção de dependências.

\subsection{Criando um projeto do Gradle no Eclipse}

O modelo de projeto usado nos exemplos desse trabalho será o Gradle Project. Esse modelo de projeto está localizado na pasta Gradle na arvore de novos projetos do Eclipse. 

\figura{gradleproject1.png}{Criando um projeto do Gradle}

É criado um projeto com e seguinte estrutura:

\figura{gradleproject1.png}{Estrutura inicial de um projeto do Gradle}

As pastas “src/main/java” e “src/main/resources” devem armazenar, respectivamente, código fonte Java e recursos utilizados na aplicação.  As pastas “src/test/java” e “src/test/resources” são utilizadas para testes unitários. As pastas de testes não serão utilizadas nos nossos exemplos e serão removidas. A pasta build é para onde irão todos os artefatos gerados pelo projeto.

O arquivo build.gradle é o arquivo de definição de projeto do Gradle. Nele podem ser criados scripts para controlar a construção de artefatos, baixar bibliotecas de terceiros, configurar testes unitários e realizar várias outras tarefas. Os scripts do Gradle são escritos em Groovy, outra linguagem compatível com a JVM.

\groovycode{code/buildgradle.txt}
\end{document}